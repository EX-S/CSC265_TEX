\documentclass[10pt]{article}
\usepackage{amsmath}
\usepackage{amssymb}
\usepackage{amsthm}
\usepackage{fullpage}
\usepackage{comment}
\usepackage{graphicx}
\usepackage{listings}
\usepackage{enumitem}
\usepackage{scrextend}

\newcommand{\iimplies}{\mbox{ IMPLIES }}
\newcommand{\oor}{\mbox{ OR }}
\newcommand{\aand}{\mbox{ AND }}
\newcommand{\nnot}{\mbox{ NOT }}
\newcommand{\iiff}{\mbox{ IFF }}
\newcommand{\xxor}{\mbox{ XOR }}
%\newcommand{\algorithmicbreak}{\textbf{break}}
%\newcommand{\Break}{\State \algorithmicbreak}

\newtheorem{theorem}{Theorem}
\newtheorem{corollary}{Corollary}[theorem]
\newtheorem{lemma}[theorem]{Lemma}

\begin{document}

\begin{center}
{\bf \Large \bf CSC265H Assignment 7}
\end{center}

\noindent
Yibin Zhao\\
1002996261\\
NO OUTSIDE DISCUSSION\\
NO EXTRA MATERIAL CONSULTED\\

\begin{comment}
	Consider the DISJOINT SET data sturcture that represents each set by a tree, 
	preform LINK(x, y) by making y a child of x, and use path compression during FIND-SET.

	For any node x, let w(x) denote the number of nodes in the subtree rooted at x.
\end{comment}

\begin{enumerate}

	\begin{comment}
	Prove that, if n MAKE-SET operations and any number of LINK and FIND-SET operations have 
	been performed, then any node has at most $\log_2(n)$ ancestor v, 
	such that $w(parent(v)) \geq 2w(v)$.
	\end{comment}
	\item
		Suppose after a sequence of those operations, there exists a node $x$
		such that $x$ has more than $\log_2(n)$ ancestors which satisfy
		$w(\text{parent}(v)) \geq 2w(v)$. 
		Let $a_1, a_2, \cdots, a_p$ be the sequence of all such ancestors of
		$x$ ordered from the lowest to the lowest. 
		$p > \log_2(n)$.
		Let $x = a_0$. 

		\textit{Claim:} $\forall i \in \{0, 1, \cdots , p\}. w(a_i) \geq 2^i$
		\begin{proof}
			Let $P(i)$ be the predicate "$w(a_i) \geq 2^i$". 
			We want to prove $\forall i \in \{0, 1, \cdots, p\}. P(i)$.
			
			\textbf{Base Case:} $i = 0$
			\begin{addmargin}[1em]{0em}
				The subtree rooted at $x$ has at least a node $x$, which means that
				$w(x) = w(a_0) \geq 1 = 2^0$.

				Thus $P(0)$ is true.
			\end{addmargin}

			\textbf{Inductive Step:}
			\begin{addmargin}[1em]{0em}
				Let $P(i)$ be true for some $i \in \{0, 1, \cdots, p-1\}$.
				Then, $w(a_i) \geq 2^i$.
				Since $a_{i+1}$ is an ancestor of $a_i$, $w(a_{i+1}) \geq
				w(\text{parent}(a_i)) \geq 2w(a_i) \geq 2^{i+1}$.

				Therefore, $P(i+1)$ is true.
			\end{addmargin}

			By induction, the claim is true. 
		\end{proof}

		By the claim $w(a_p) \geq 2^{p} > 2^{\log_2 (n)} = n$.
		This is a contradiction, since there could be at most $n$ nodes in a
		tree.

		Hence, the proposition  holds.


	\begin{comment}
	Using the accounting method, prove that the amartized complexity of MAKE-SET is $O(1)$ 
	and the amartized complexity of LINK and FIND-SET are $O(\log(n))$, 
	where $n$ is the number of MAKE-SET operations performed.

	Consider the following credit invariant: 
		Each node $x$ has $\log_2(w(x))$ credits associated with it. 
	\end{comment}
	\item
		Since there are only $n$ MAKE-SET operations, the number of LINK
		operation is at most $n-1$.
		Let the number of pointer assignments be the cost of these algorithms. 
		
		The actual cost of each operation is 
		\begin{align*}
			&\text{MAKE-SET}	&&  1 \\
			&\text{LINK}		&&  1 \\
			&\text{FIND-SET}	&&  h(x)+1
		\end{align*}
		For operation FIND-SET, $h(x)$ is the height of the node $x$ to be
		performed on.
		Since the operation recusively find the root and assign the pointer to
		the root back to those nodes, $h(x)$ operations are needed.
		Notice that the root is of height 0, 1 extra operation is needed for the
		root.
		Thus, FIND-SET has an actual cost of $h(x)+1$.

		Consider the credit invariant: Each node $x$ has $\log_2w(x)$ credits
		associated with it.

		We start with a disjoint set with no elements in it.
		Since there is no nodes in the disjoint set initially, the credit
		invariant is trivial.


		Now suppose credit invariant holds before some operation.

		For a MAKE-SET operation, charge $\$1$ for the actual cost.
		All other sets except the new set are unchanged.
		Since there is only one element in the new set, $\log_2w(x) = \log_21 =
		0$, and thus, no new credits.
		Therefore, the total charge of a MAKE-SET operation is $\$1$, which is
		in $O(1)$.
		Also, the credit invariant holds for a MAKE-SET operation.


		Consider a LINK operation LINK($x, y$).
		$\$1$ is charged to pay for the actual cost of this operation.
		Since $x$ and $y$ are the representatives for their sets, $w(x)$ and
		$w(y)$ are the number of nodes of those sets respectively. 
		By the implementation of LINK, only the credits on node $x$ needs to be
		changed in order to maintain the credit invarant.
		The new set consists of $w(x)+w(y)$ elements, which implies that the
		new credit on $x$ is $\log_2(w(x)+w(y))$.
		Then, extra $\$\log_2(w(x)+w(y)) - \log_2w(x) = \$\log_2(1+w(y)/w(x))$
		needs to be charged in order to maintain the credit invariant. 

		Suppose LINK has been performed $i$ times, where $i \in \{0, 1, \cdots
		n-2\}$, then the largest set is of order $i+1$.
		Then, consider a new LINK operation.
		$\log_2(1+w(y)/w(x)) \leq \log_2(1+(i+1)/1) = \log_2(i+2)$.
		Therefore the amartized cost of a LINK operation is at most
		$$\$\log_2(n-2+2)+1 = \log_2n+1 \in O(\log n)$$


		Now, consider a FIND-SET operation.
		Suppose this FIND-SET operation is performed on a node $x$.
		Then, $\$h(x)+1$ is charged for this operation.
		Notice that the set where $w$ belongs has at most $n$ nodes.
		Then, by 1. there are at most $\log_2n$ ancestors $v$ of $x$ such that
		$w(\text{parent}(v)) \geq 2w(v)$.
		This implies that there are at least $h(x)-1 - \log_2n$ ancestors $v$
		of $x$ such that $w(\text{parent}(v)) < 2w(v)$ (Note that the root
		cannot be such an ancestor).
		Let $x_0, x_1, \cdots, x_{h(x)}$ be the sequence of nodes from the root
		to $x$.
		To keep the credit invariant, $\$\log_2w(x_i) - \log_2(w(x_i) -
		w(x_{i+1}))$ needs to be withdrawn from node $x_i$, where $i \in \{1,
		2, \cdots, h(x)-1\}$. 
		Note that the credit on $x_{h(x)} = x$ need not be changed, since the
		subtree rooted at $x$ is unchanged.
		For any nodes other than $x_0, x_1, \cdots, x_{h(x)}$, the subtrees
		rooted at them are also unchanged, and thus the credit invarient holds
		for them.

		Consider the total number of credit withdrawn from the bank:
		\begin{align*}
			& \sum_{i=1}^{h(x)-1}(\log_2w(x_i) - \log_2(w(x_i) - w(x_{i+1}))) \\
			=& \sum_{i=1}^{h(x)-1} \left(- \log_2(1 -
			\frac{w(x_{i+1})}{w(x_i)}) \right) \\
		\end{align*}
		Consider each $- \log_2(1 - \frac{w(x_{i+1})}{w(x_i)})$. \\
		\textit{Case 1:} $w(x_{i}) \geq 2w(x_{i+1})$
		\begin{addmargin}[1em]{0em}
			Then, $0 < \frac{w(x_{i+1})}{w(x_i)} \leq \frac{1}{2}$.
			This implies that $0< -\log_2(1-\frac{w(x_{i+1})}{w(x_i)}) \leq 1$.
		\end{addmargin}
		\textit{Case 2:} $w(x_{i}) < 2w(x_{i+1})$
		\begin{addmargin}[1em]{0em}
			Then, $frac{w(x_{i+1})}{w(x_i)} > \frac{1}{2}$, and thus, $-
			\log_2(1 - \frac{w(x_{i+1})}{w(x_i)}) > 1$
		\end{addmargin}
		Since there are at most $\log_2n$ of case 1 and at least
		$h(x)-1-\log_2n$ of case 2, the total number of credit withdrawn from
		the bank is at least
		\begin{align*}
			& \sum_{i=1}^{h(x)-1} \left(- \log_2(1 - \frac{w(x_{i+1})}{w(x_i)})
			\right) \\
			>& \log_2n \cdot 0 + (h(x) - 1 - \log_2n) \cdot 1 \\
			=& h(x) - 1 - \log_2n && (h(x) > \log_2n)
		\end{align*}
		When, $\log_2n \geq h(x)$, the total credit is at least 0.

		Then, the amartized cost is at most $\$(h(x)+1) - (h(x)-1-\log_2n) =
		\$\log_2n + 2$ when $h(x) > \log_2n$, and $\$(h(x)+1)-0 = \$h(x)+1$
		when $h(x) \leq \log_2n$.
		Either case, the amartized cost is always in $O(\log n)$.
		
		Also, the credit invariant is true for all nodes after FIND-SET
		operation. 

		
		Therefore, the credit invariant also holds after a operation.

		By induction, the credit invariant is true.
		Also, the armartized time complexity of MAKE-SET, LINK and FIND-SET are
		$O(1)$, $O(\log n)$ and $O(\log n)$ respectively for this sequence of
		operations.

	\begin{comment}
	Using the potential method, prove that the amartized complexity of MAKE-SET is $O(1)$
	and the amartized complexity of LINK and FIND-SET are $O(\log(n))$, where $n$ is the number of
	MAKE-SET operations that have been performed.
	\end{comment}
	\item

\end{enumerate}

\end{document}

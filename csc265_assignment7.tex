\documentclass[10pt]{article}
\usepackage{amsmath}
\usepackage{amssymb}
\usepackage{amsthm}
\usepackage{fullpage}
\usepackage{comment}
\usepackage{graphicx}
\usepackage{listings}
\usepackage{enumitem}
\usepackage{scrextend}

\newcommand{\iimplies}{\mbox{ IMPLIES }}
\newcommand{\oor}{\mbox{ OR }}
\newcommand{\aand}{\mbox{ AND }}
\newcommand{\nnot}{\mbox{ NOT }}
\newcommand{\iiff}{\mbox{ IFF }}
\newcommand{\xxor}{\mbox{ XOR }}
%\newcommand{\algorithmicbreak}{\textbf{break}}
%\newcommand{\Break}{\State \algorithmicbreak}

\newtheorem{theorem}{Theorem}
\newtheorem{corollary}{Corollary}[theorem]
\newtheorem{lemma}[theorem]{Lemma}

\begin{document}

\begin{center}
{\bf \Large \bf CSC265H Assignment 7}
\end{center}

\noindent
Yibin Zhao\\
1002996261\\
NO OUTSIDE DISCUSSION\\
NO EXTRA MATERIAL CONSULTED\\

\begin{comment}
	Consider the DISJOINT SET data sturcture that represents each set by a tree, 
	preform LINK(x, y) by making y a child of x, and use path compression during FIND-SET.

	For any node x, let w(x) denote the number of nodes in the subtree rooted at x.
\end{comment}

\begin{enumerate}

	\begin{comment}
	Prove that, if n MAKE-SET operations and any number of LINK and FIND-SET operations have 
	been performed, then any node has at most $\log_2(n)$ ancestor v, 
	such that $w(parent(v)) \geq 2w(v)$.
	\end{comment}
	\item
		Suppose after a sequence of those operations, there exists a node $x$
		such that $x$ has more than $\log_2(n)$ ancestors which satisfy
		$w(\text{parent}(v)) \geq 2w(v)$. 
		Let $a_1, a_2, \cdots, a_p$ be the sequence of all such ancestors of
		$x$ ordered from the lowest to the lowest. 
		$p > \log_2(n)$.
		Let $x = a_0$. 

		\textit{Claim:} $\forall i \in \{0, 1, \cdots , p\}. w(a_i) \geq 2^i$
		\begin{proof}
			Let $P(i)$ be the predicate "$w(a_i) \geq 2^i$". 
			We want to prove $\forall i \in \{0, 1, \cdots, p\}. P(i)$.
			
			\textbf{Base Case:} $i = 0$
			\begin{addmargin}[1em]{0em}
				The subtree rooted at $x$ has at least a node $x$, which means that
				$w(x) = w(a_0) \geq 1 = 2^0$.

				Thus $P(0)$ is true.
			\end{addmargin}

			\textbf{Inductive Step:}
			\begin{addmargin}[1em]{0em}
				Let $P(i)$ be true for some $i \in \{0, 1, \cdots, p-1\}$.
				Then, $w(a_i) \geq 2^i$.
				Since $a_{i+1}$ is an ancestor of $a_i$, $w(a_{i+1}) \geq
				w(\text{parent}(a_i)) \geq 2w(a_i) \geq 2^{i+1}$.

				Therefore, $P(i+1)$ is true.
			\end{addmargin}

			By induction, the claim is true. 
		\end{proof}

		By the claim $w(a_p) \geq 2^{p} > 2^{\log_2 (n)} = n$.
		This is a contradiction, since there could be at most $n$ nodes in a
		tree.

		Hence, the proposition  holds.


	\begin{comment}
	Using the accounting method, prove that the amartized complexity of MAKE-SET is $O(1)$ 
	and the amartized complexity of LINK and FIND-SET are $O(\log(n))$, 
	where $n$ is the number of MAKE-SET operations performed.

	Consider the following credit invariant: 
		Each node $x$ has $\log_2(w(x))$ credits associated with it. 
	\end{comment}
	\item

	\begin{comment}
	Using the potential method, prove that the amartized complexity of MAKE-SET is $O(1)$
	and the amartized complexity of LINK and FIND-SET are $O(\log(n))$, where $n$ is the number of
	MAKE-SET operations that have been performed.
	\end{comment}
	\item

\end{enumerate}

\end{document}

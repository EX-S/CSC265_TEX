\documentclass[10pt]{article}
\usepackage{amsmath}
\usepackage{amssymb}
\usepackage{amsthm}
\usepackage{fullpage}
\usepackage{comment}
\usepackage{graphicx}
\usepackage{listings}
\usepackage{enumerate}
\usepackage{scrextend}
\usepackage{algorithm}
\usepackage{algpseudocode}

\newcommand{\iimplies}{\mbox{ IMPLIES }}
\newcommand{\oor}{\mbox{ OR }}
\newcommand{\aand}{\mbox{ AND }}
\newcommand{\nnot}{\mbox{ NOT }}
\newcommand{\iiff}{\mbox{ IFF }}
\newcommand{\xxor}{\mbox{ XOR }}

\begin{document}

\begin{center}
{\bf \Large \bf CSC265H Assignment 2}\\
\end{center}

\noindent
Yibin Zhao\\
1002996261\\
NO OUTSIDE DISCUSSION\\
NO EXTRA MATERIAL CONSULTED\\

\begin{enumerate}

	\item We will start the proof by proving some lemmas. 
	\textbf{Lemma 1:} A subtree of an R-tree is also an R-tree.
	\begin{proof}
		Let $S$ be any subtree of an R-tree $T$. \\
		\texit{Case 1:} $S$ is empty.
		\begin{addmargin}[1em]{0em}
			This is trivial, since an empty can also be an R-tree.
		\end{addmargin}
		\textit{Case 2:} $S$ has only 1 node.
		\begin{addmargin}[1em]{0em}
			The only node is the root of the subtree $S$. \\
			Let the node be $x$. \\
			By properties of R-tree, the weight of $x$ must be a positive
			integer. \\
			Since $S$ is a subtree of $T$, $x$ is the a leaf in $T$. \\
			Then $x$ must have a positive integer weight. \\
			Since there existes only one path in $S$, the third property will
			always be satisfied. \\
			Therefore, $S$ is an R-tree.
		\end{addmargin}
		\textit{Case 3:} $S$ has more than 1 node. \\
		\begin{addmargin}[1em]{0em}
			Since $S$ is a subtree of $T$, all internal nodes in $S$ are
			internal nodes in $T$, and all leaves in $S$ are leaves in $T$. \\
			Then the first and second properties of R-tree are automatically
			satisfied. \\
			Let $x$ be the root of $S$. \\
			Since any path from the root to a leaf in $T$ have the same weight,
			all paths from the root to a leaf through $x$ have the same weight
			as well. \\
			Let the weight be $w$. \\
			There is only one path from the root of $T$ to $x$. \\
			Let the weight of this path be $w'$. \\
			Suppose $x$ has a weight of $w''$. \\
			Then, any path from $x$ to a leaf will have the weight of
			$w-w'+w''$. \\
			That is, the third property is also satisfied. \\
			Hence, $S$ is an R-tree.
		\end{addmargin}
		From all these cases, we can conclude that $S$ is an R-tree. \\
		By generalization, a subtree of an R-tree is an R-tree. 
	\end{proof}

	\texbf{Lemma 2:} Suppose there is an R-tree $T$ with an non-empty subtree
		$S$. Let $S'$ be an R-tree such that the weight of all path from the
		root to a leaf have the same weight as in $S$. Replace $S$ by $S'$ and
		obtain a new tree $T'$. Then $T'$ is an R-tree. 
	\begin{proof}
		Let $w$ be the weight of all path from the root of $T$ to a leaf. \\
		Let $w'$ be the weight of all path from the root of $S$ to a leaf. \\
		Then $w'$ is also the weight of all path from the root of $S'$ to a
		leaf. \\
		Since $S'$ is an R-tree, all nodes in $S'$ satisfy the first two
		properties. \\
		Similarly all nodes in $T$ satisfy the first two properties. \\
		Then all unreplaced nodes in $T'$ satisfy the first two properties. \\
		Since $S'$ is a subtree of $T'$, all internal nodes of $S'$ are
		internal nodes of $T'$. \\
		Similarly, all leaves of $S'$ are leaves on $T'$ \\
		Therefore, $T'$ satisfy the fiest two properties of an R-tree. \\
		Let the root of $S$ and $S'$ be $x$ and $x'$ respectively. \\
		As proven in Lemma 1, all path from the root of $T$ to a leaf through
		$x$ have the same sum of weights, which is also $w$. \\
		Consider a path from the root of $T'$ to a leaf. \\
		There are two possible cases:  \\
		(1) The path includes $x'$. \\
		(2) The path does not include $x'$. \\
		Consider the path from case (1). 
		\begin{addmargin}[1em]{0em}
			The total weight of this path would be $w$, since there always
			exists an identical path in $T$. 
		\end{addmargin}
		Then, consider the path from case (2)
		\begin{addmargin}[1em]{0em}
			Let $u$ and $u'$ be the weight of $x$ and $x'$ respectivly. \\
			Then the only path from the root of $T$ to $x$ has a weight of
			$w-w'+u$. \\
			Consider the path from the root of $T$ to $x'$. \\
			This path is almost identical to the previous one, except we
			replace $x$ by $x'$, in which case, the path has a weight of
			$w-w'+u'$. \\
			Since every path from $x'$ to a leaf have the weight of $w'$, the
			total weight of a path from the root of $T'$ to a leaf through $x'$
			would have a weight of $w-w'+u' - u' + w' = w$. \\
		\end{addmargin}
		Thus, any path from the root of $T'$ to a leaf have a weight of $w$. \\
		Therefore, $T'$ is an R-tree.
	\end{proof}

	From Lemma 1 and Lemma2, suppose we want to prove that after some operation
	on an R-tree, the result is still an R-tree, it is sufficient to prove that
	the subtree to be replaced and the new subtree satisfy the conditions in
	Lemma 2.
	
	\begin{proof}
	\textbf{INSERT}
	\begin{addmargin}[1em]{0em}
		\textit{Case 1:} Inserting to an empty tree.
		\begin{addmargin}[1em]{0em}
			Simply add a root to the tree and make the key equal to the element to be inserted, also, weight it 1. \\
			There is no non-leaf node in the tree. \\
			The root is a leaf and it has a positve integer weight. \\
			There is only one path from the root to a leaf (the root), so the sum of the weights of the nodes on any path is the same.
			Therefore, it is a R-tree.
		\end{addmargin}
		\textit{Case 2:} Inserting into an arbitrary non-empty R-tree.
		\begin{addmargin}[1em]{0em}
			Suppose the insertion is performed on a leaf $x$, weighted $w$. \\
			By the properties of R-tree, $w$ is a positive integer. \\
			The insertion give both new leaves weight 1, and thus, the property "each leaf has a positive integer weight" is satisfied. \\
			The new internal node is weighted $w-1$. \\
			$w-1$ is non-negative, since $w$ is a positive integer. \\
			Then, in the new tree, each non-leaf node has a non-integer weight. \\
			Also, each path in the subtree has a weight of $w$, which implies that the two new paths have the same weights as the path from root to $x$ on the original tree.
			Therefore, all path from the root to a leaf have the same weight.
			Hence, the new tree is a R-tree, since all three properties are satisfied.
		\end{addmargin}

		In conclusion, if an insertion is applied on a R-tree, the result is a R-tree.
	\end{addmargin}

	\textbf{DELETE}
	\begin{addmargin}[1em]{0em}
		\textit{Case 1:} Deleting from a R-tree with one node.
		\begin{addmargin}[1em]{0em}
			Simply delete the root of the tree and it becomes am empty R-tree.
		\end{addmargin}
		\textit{Case 2:} Deleting from a R-tree with more than one node.
		\begin{addmargin}[1em]{0em}
			Since R-tree is leaf-oriented, a deletion will always be performed on a leaf. \\
			Suppose the leaf to be performed on is $x$ with weight $w'$. \\
			Since the R-tree has more than one node, which means that $x$ must have a parent. \\
			Let the parent be $p$ with weight $w$. \\
			Since $p$ is a non-leaf node, it has two children. \\
			That is, $x$ has a sibling $y$. \\
			Let $w''$ be the weight of $y$. \\
			Each path from a leaf upto $p$ has a weight of $w+w'$. \\
			Then, $w' \geq w''$. \\
			$y$ might be a leaf or an internal node. \\
			Either case, all other leaves except $x$ will remain the same after the deletion, and thus, their weights are all positive. \\
			$x$ is a leaf, so $w'$ is positive, and $w+w'$ will also be postive. \\
			Thus, the weight of $y$ will always be a positive integer. \\
			Therefore, all leaves have positive weights.
			Similarly, all other internal nodes except $p$ remain the same after the deletion. \\
			Then, all non-leaf node have non-negative weight.\\
			Since any path from $p$ to a leaf has a weight of $w+w'$, any path from $y$ to a leaf has a weight of $w'$. \\
			Consider the new tree after deletion. \\
			$x$ and $p$ are removed and $y$ replaced $p$. \\
			The path from $y$ to a leaf of the subtree all have the same weight of $w+w''+w'-w'' = w+w'$, which is the same as the weigh from $p$ to a leaf on the original tree. \\
			Hence, all path from a leaf to the root has the same weight.
		\end{addmargin}
		Therefore, if an deletion rule is applied to an R-tree, the result is an R-tree.	
	\end{addmargin}

	\textbf{REBALANCING}
	\begin{addmargin}[1em]{0em}
		\textit{Rule 1:}
		\begin{addmargin}
			The initial R-tree have a root which is not a leaf. \\
			Changing the weight on the root won't affect the weight of any leaf
			or any other internal node. \\
			Also, 1 is a positive integer. \\
			So, the first two properties of an R-tree is satisfied. \\
			Since the weight of all path from the root to a leaf will always
			include the weight of the root, modifying the weight of the root
			wont affect the third property of an R-tree. \\
			Therefore, all properties are satisfied, and thus, the new tree is
			also an R-tree. \\
		\end{addmargin}
		\textit{Rule 2:}
		\begin{addmargin}
			Consider the subtree $t$ where $x$ is the root. \\
			$x$ is an internal node, since it has two children. \\
			Since $w \geq 1$, $w-1 \geq 0$ which is legal for an internal node.\\
			Also 1 is a totally legal weight for any node. \\
			Let $u$ and $v$ be the children of $x$, and let $c$ be the child of
			$u$ with a weight of 0. \\
			Let weight for any path from $x$ to a leaf be $w+k$, where $k$ is
			some positive integer. \\
			Then all path for the new subtree will have a weight of $w-1 + 1 +
			(w+k) - w = w+k$, which is the same as the initial subtree. \\
			Therefore, the new tree is an R-tree after applying rule 2.
		\end{addmargin}

		\textit{Rule 3:}
		%TODO: Rule 3 (a) and (b)

		%TODO: Rule 4
		\texit{Rule 4:}
		\begin{addmargin}[1em]{0em}
		\end{addmargin}

		%TODO
		\texit{Rule 5:}
		\begin{addmargin}[1em]{0em}
		\end{addmargin}

		%TODO
		\texit{Rule 6:}
		\begin{addmargin}[1em]{0em}
		\end{addmargin}

		\texit{Rule 7:}
		%TODO: Rule 7 (a) and (b)

		%TODO
        \texit{Rule 8:}
        \begin{addmargin}[1em]{0em}
		\end{addmargin}

	\end{addmargin}
	
	\end{proof}

	\item
	%TODO: no violation
	To prove that any R-tree can be tranformed by a series of rebalancing rules into an R-tree with no violation, we need to prove the following two propositions: \\
	(1) For any violation within an R-tree, there always existes a rebalancing rule that could be applied to this violation. \\
	(2) Each rebalacing rule can reduce one violation without causing other violations. \\

	Before start working on (1) and (2), let's first prove some Lemmas to help us. 

	\textbf{Lemma 3:} Consider an R-tree with 0-0 violation, Then there always exists an $x$ and its parent $p$, both weighted 0 such that $p$ is the root or $p$ has a positivly weighted parent $g$. \\
	\begin{proof}
	Let $T$ be an R-tree that have 0-0 violation. \\
	Suppose there does not exists such $x$ and $p$. \\
	Then $p$ must have a parent $g$, and $g$ must have 0 weight. \\
	Let ($x_1$, $x_2$, \cdots, $x_{n-1}$, $x_{n}$) be a path from the root $x_1$ to a 0-0 violation, $x_{n}$ and its parent $x_{n-1}$. \\
	Both $x_{n-1}$ and $x_{n}$ are weighted 0. \\
	Suppose, there exists an $i \in \{1, 2, \cdots, n\}$ such that $x_{i}$ has a positive  weight. \\
	Then, $x_{i+1}$ and $x_{i+2}$ cannot be both 0 weighted if they exists. \\
	Therefore, for any $j \geq i \aand j+2 \leq n$, at leasts two of $x_{j}, x_{j+1}, x_{j+2}$ are weighted positively. \\
	$x_{n-2}$ must exists, or otherwise, $x_{n-1}$ is the root. \\
	Then, in $x_{n-2}, x_{n-1}$ and $x_{n}$, there must be at least two of them not weighted 0. \\
	However, $x_{n-1}$ and $x_{n}$ both have 0 weight. \\
	This is a contradiction. \\
	Therefore, by generalization, for any R-tree with 0-0 violation, such $x$ and $p$ in Lemma 3 always exist. 
	\end{proof}

	(1)
	\begin{proof}
	First, consider a 0-0 violation in an arbitrary R-tree, say $T$. \\
	Let $x$ be a node with weight 0, and let $p$ be its parent whose weight is also 0. \\
	Then, $x$ is not a leaf. \\
	Also, $x$ must have a sibling, call it $y$. \\
	\textit{Case 1:}$p$ is the root
	\begin{addmargin}[1em]{0em}
		Then Rule 1 could be applied.
	\end{addmargin}
	\textit{Case 2:}$p$ has a parent $g$
	\begin{addmargin}[1em]{0em}
		Then $p$ must also have a sibling, call it $q$. 
		\textit{Case a:} $g$ has an positive integer weight
		\begin{addmargin}[1em]{0em}
			If $q$ has a 0 weight, then Rule 2 could be applied. \\
			It $q$ has a positive integer weight, then Rule 3 could be applied.
		\end{addmargin}
		\textit{Case b:} $g$ has 0 weight
		\begin{addmargin}[1em]{0em}
			By Lemma 3, we don't need to consider this case, since we can always solve other cases if there exists a 0-0 violation.
		\end{addmargin}
	\end{addmargin}
	\end{proof}

	%TODO: (2)

\end{enumerate}

\end{document}

\documentclass[10pt]{article}
\usepackage{amsmath}
\usepackage{amssymb}
\usepackage{amsthm}
\usepackage{fullpage}
\usepackage{comment}
\usepackage{graphicx}
\usepackage{listings}
\usepackage{enumerate}
\usepackage{scrextend}
\usepackage{algorithm}
\usepackage{algpseudocode}

\newcommand{\iimplies}{\mbox{ IMPLIES }}
\newcommand{\oor}{\mbox{ OR }}
\newcommand{\aand}{\mbox{ AND }}
\newcommand{\nnot}{\mbox{ NOT }}
\newcommand{\iiff}{\mbox{ IFF }}
\newcommand{\xxor}{\mbox{ XOR }}

\begin{document}

\begin{center}
{\bf \Large \bf CSC265H Assignment 2}\\
\end{center}

\noindent
Yibin Zhao\\
1002996261\\
NO OUTSIDE DISCUSSION\\
NO EXTRA MATERIAL CONSULTED\\

\begin{enumerate}

	\item
	\textbf{INSERT}
	\begin{addmargin}[1em]{0em}
		\textit{Case 1:} Inserting to an empty tree.
		\begin{addmargin}[1em]{0em}
			Simply add a root to the tree and make the key equal to the element to be inserted, also, weight it 1. \\
			There is no non-leaf node in the tree. \\
			The root is a leaf and it has a positve integer weight. \\
			There is only one path from the root to a leaf (the root), so the sum of the weights of the nodes on any path is the same.
			Therefore, it is a R-tree.
		\end{addmargin}
		\textit{Case 2:} Inserting into an arbitrary non-empty R-tree.
		\begin{addmargin}[1em]{0em}
			Suppose the insertion is performed on a leaf $x$, weighted $w$. \\
			By the properties of R-tree, $w$ is a positive integer. \\
			The insertion give both new leaves weight 1, and thus, the property "each leaf has a positive integer weight" is satisfied. \\
			The new internal node is weighted $w-1$. \\
			$w-1$ is non-negative, since $w$ is a positive integer. \\
			Then, in the new tree, each non-leaf node has a non-integer weight. \\
			Also, each path in the subtree has a weight of $w$, which implies that the two new paths have the same weights as the path from root to $x$ on the original tree.
			Therefore, all path from the root to a leaf have the same weight.
			Hence, the new tree is a R-tree, since all three properties are satisfied.
		\end{addmargin}

		In conclusion, if an insertion is applied on a R-tree, the result is a R-tree.
	\end{addmargin}

	\textbf{DELETE}
	\begin{addmargin}[1em]{0em}
		\textit{Case 1:} Deleting from a R-tree with one node.
		\begin{addmargin}[1em]{0em}
			Simply delete the root of the tree and it becomes am empty R-tree.
		\end{addmargin}
		\textit{Case 2:} Deleting from a R-tree with more than one node.
		\begin{addmargin}[1em]{0em}
			Since R-tree is leaf-oriented, a deletion will always be performed on a leaf. \\
			Suppose the leaf to be performed on is $x$ with weight $w'$. \\
			Since the R-tree has more than one node, which means that $x$ must have a parent. \\
			Let the parent be $p$ with weight $w$. \\
			Since $p$ is a non-leaf node, it has two children. \\
			That is, $x$ has a sibling $y$. \\
			Let $w''$ be the weight of $y$. \\
			Each path from a leaf upto $p$ has a weight of $w+w'$. \\
			Then, $w' \geq w''$. \\
			$y$ might be a leaf or an internal node. \\
			Either case, all other leaves except $x$ will remain the same after the deletion, and thus, their weights are all positive. \\
			$x$ is a leaf, so $w'$ is positive, and $w+w'$ will also be postive. \\
			Thus, the weight of $y$ will always be a positive integer. \\
			Therefore, all leaves have positive weights.
			Similarly, all other internal nodes except $p$ remain the same after the deletion. \\
			Then, all non-leaf node have non-negative weight.\\
			Since any path from $p$ to a leaf has a weight of $w+w'$, any path from $y$ to a leaf has a weight of $w'$. \\
			Consider the new tree after deletion. \\
			$x$ and $p$ are removed and $y$ replaced $p$. \\
			The path from $y$ to a leaf of the subtree all have the same weight of $w+w''+w'-w'' = w+w'$, which is the same as the weigh from $p$ to a leaf on the original tree. \\
			Hence, all path from a leaf to the root has the same weight.
		\end{addmargin}
		Therefore, if an deletion rule is applied to an R-tree, the result is an R-tree.	
	\end{addmargin}

	%TODO: Rotation
	\textbf{ROTATION}
	\begin{addmargin}[1em]{0em}
	\end{addmargin}

	\item
	%TODO: no violation

\end{enumerate}

\end{document}

\documentclass[10pt]{article}
\usepackage{amsmath}
\usepackage{amssymb}
\usepackage{amsthm}
\usepackage{fullpage}
\usepackage{comment}
\usepackage{graphicx}
\usepackage{listings}
\usepackage{enumitem}
\usepackage{scrextend}

\newcommand{\iimplies}{\mbox{ IMPLIES }}
\newcommand{\oor}{\mbox{ OR }}
\newcommand{\aand}{\mbox{ AND }}
\newcommand{\nnot}{\mbox{ NOT }}
\newcommand{\iiff}{\mbox{ IFF }}
\newcommand{\xxor}{\mbox{ XOR }}
%\newcommand{\algorithmicbreak}{\textbf{break}}
%\newcommand{\Break}{\State \algorithmicbreak}

\newtheorem{theorem}{Theorem}
\newtheorem{corollary}{Corollary}[theorem]
\newtheorem{lemma}[theorem]{Lemma}

\begin{document}

\begin{center}
{\bf \Large \bf CSC265H Assignment 6}
\end{center}

\noindent
Yibin Zhao\\
1002996261\\
NO OUTSIDE DISCUSSION\\
NO EXTRA MATERIAL CONSULTED\\

\begin{comment}
The median of a set of numbers is the element with rank $\lfloor (n+1)/2 \rfloor$.
\end{comment}

\begin{enumerate}
	
	\begin{comment}
	Prove that any decesion tree using 2-way comparisons that finds the median of $n$ distinct numbers 
	also determines the set of numbers greater than the median and the set of numbers smaller than the median.
	\end{comment}

	\item
	Since a 2-way comparison between $A[i]$ and itself always returns the same
	result regardless of the value of $A[i]$, we only need to consider
	comparisons between two elements with different indices. 

	\begin{comment}
	\begin{lemma}
	For any input of size $n$, any number is compared by at least once in its path
	of a decision tree that solve this problem.
	\end{lemma}

	\begin{proof}
	Consider an array $A$ of size $n$. 
	Suppose there exists such an element $A[j]$ not being compared by any other elements.
	Let $A[k]$ be the number with rank $\lfloor (n+1)/2 \rfloor$.
	Then, the decision tree gives an output of $A[k]$ for this input $A$. 

	Let $B[i] = A[i]$ for all $i \neq j$. \\
	If $A[j] > A[k]$, then set $B[j] = A[k]-1$. 
	Otherwise, set $B[j] = A[k]+1$. \\
	This guarantee that the median of $B$ is not $B[k]$.
	
	Consider any comparison in the path of the decision tree for input $A$.
	Denote the two numbers being compared as $A[p]$ and $A[q]$, where $p \neq q$. \\
	By the assumption, neither $p, q$ could be $j$. 
 	Then the same comparison between $B[p]$ and $B[q]$ should have the same result. \\
	Thus, the output is consistent for $A$ and $B$.
	
	However, the output for $B$ is not correct. 
	This is a contradiction.
	
	Therefore, the lemma is true.
	\end{proof}
	\end{comment}

	\begin{lemma}
	For any input of size $n$, let $S_1, S_2$ be a non-empty partition of the set of
	all numbers $S$. 
	Then there exists elements $e_1 \in S_1$ and $e_2 \in S_2$ such that there
	is a comparison between them in the path of this input on a decision tree.
	\end{lemma}

	\begin{proof}
		Consider an input $A[1 \ldots n]$.
		Let $S$ be the set of all numbers in $A$.
		Suppose $S_1, S_2$ is a non-empty partition of $S$ such that $\forall e_1
		\in S_1. \forall e_2 \in S_2.$ there is no comparison between $e_1$ and
		$e_2$.
	
		Without loss of generality, assume that the median $A[k]$ is in $S_1$.
		Let $A[u_1], \cdots, A[u_p]$ be all elements in $S_1$ and $A[v_1],
		\cdots, A[v_q]$ be all elements in $S_2$. 
		Let $B[u_i] = A[u_i]$ for all $i \in \{1, \ldots, p\}$. \\
		If there exists a $A[v_j] < A[k]$, then let $B[v_i] = A[v_i] + c$ for
		all $i \in \{1, \ldots, q\}$, where constant $c$ satisfy $A[v_j] + c >
		A[k]$. \\
		Otherwise, let $B[v_i] = A[v_i] - c$ for all possible $i$, where $c \in
		\mathbb{R}$ is some constant that satisfy there is a $A[v_i]$ such that
		$A[v_i] - c < A[k]$. 

		This guarantee the median of $B$ is not $B[k]$

		Consider a comparison between $A[i]$ and $A[j]$, where $i \neq j$.
		Then either both are in $S_1$ or both are in $S_2$. \\
		If both in $S_1$, then $B[i] = A[i]$ and $B[j] = A[j]$, which implies
		that the result should be the same. \\
		Otherwise, since $B[j] - B[i] = A[j] - A[i]$, the result of the same
		comparison between $B[i]$ and $B[j]$ should be the same as the one for
		$A$. 

		Thus, the output is consistent for $A$ and $B$. 
		However the output is wrong for $B$.

		Therefore, the lemma is true.
	\end{proof}

	Suppose there is some decision tree $T$ that solves this problem of size
	$n$ and some input $A$ of size $n$ such that $T$ cannot determine those two
	sets.

	Then there exists some element in $A$ that cannot be determined as smaller
	than or greater than the median.

	Let the set of all those elements be $S_2$, and let $S_1$ be the complement set
	on the set $S$ of all numbers in $A$.
	Notice that $S_1$ is non-empty since the median is in it.

	By the lemma, there exists a number $A[k] \in S_2$, such that $A[k]$ is
	compared with some element in $S_1$.

	Let the median be $A[j]$.
	Then, $A[k]$ cannot be compared with $A[j]$, or otherwise the relationship
	between $A[k]$ and the median is determined.
	Also, $A[k]$ cannot be compared with some number in $S_1$ whose value lies
	between $A[k]$ and $A[j]$, or otherwise the relationship between $A[k]$ and
	the median could also be determined.
	Hence, the number being compared ($A[i] \in S_1$) must satisfy $A[i] >
	\max\{A[j], A[k]\} \oor A[i] < \min\{A[j], A[k]\}$.
	Thus, by all these comparsions with $A[j]$, there is a smallest open
	interval $(a, b)$ of possible range of $A[j]$.
	$(a, b)$ must satisfy $A[k] \in (a, b)$.
	Note that $a$ could be $-\infty$ and $b$ could be $\infty$.

	Also, a smallest open interval $(c, d)$ of possible range for $A[j]$ can be determine by the
	comparisons between $A[j]$ and those elements in $S_2$. 
	Note that $c$ could be $-\infty$ and $b$ could be $\infty$.
	Similarly, $A[k] \in (c, d)$, or otherwise the relationship between $A[k]$
	and $A[j]$ could be determined.

	Thus we can obtain a smallest possible range $(a, b) \cap (c, d)$, and
	$A[k] \in (a, b) \cap (c, d)$. 
	Let $l = \max\{a, c\}$ and $r = \min\{b, d\}$, then $(a, b) \cap (c, d) =
	(l, r)$. 

	Let $B[i] = A[i]$ for all $i \neq j$. \\
	If $A[k] > A[j]$, then set $l < B[j] < A[k]$. \\
	Otherwise, set $A[k]< B[j] < r$.
	This guarantee that the median of $B$ is not $B[k]$.

	Consider a comparison between $A[u]$ and $A[v]$, where $u \neq v$. \\
	If neither $u, v$ equals to $j$, then the same comparison between $B[u]$
	and $B[v]$ has the same result. \\
	Otherwise, the result is also consistent as long as $B[j] \in (l, r)$. 

	Therefore, the output is consistent for $A$ and $B$.
	
	However, the output for input $B$ is incorrent. 
	This is a contradiction. 
	Therefore there does not exists such tree $T$. 

	Hence any decision tree using 2-way comparisons that finds the median of n
	distinct numbers also determines the set of numbers greater than the median
	and the set of numbers less than the median.
	% 1 ends here

	\begin{comment}
	Prove and $\Omega(n)$ lower bound on the worst case complexity of finding the median of $n$ numbers 
	using the decision tree with 2-way comparisons.
	\end{comment}

	\item By 1, we can make the output of the decision tree be the index of the median and the set of indecies of elements greater than the median and the set of indecies for smaller ones. \\
	Then, the total number of possible output for an array of size $n$ is
	\begin{align*}
		& n \cdot 
		\begin{pmatrix}
			n-1 \\
			\lfloor (n+1)/2 \rfloor - 1
		\end{pmatrix} \\
		=& \frac{n^{(\lfloor (n+1)/2 \rfloor)}}{\lfloor (n-1)/2 \rfloor!} \\
		\geq& \frac{n^{(\lfloor n/2 \rfloor)}}{\lfloor n/2 \rfloor!} \\
		=& \prod_{i=1}^{\lfloor n/2 \rfloor}\frac{n-i+1}{\lfloor n/2 \rfloor -i+1} \\
		\geq& 2^{\lfloor n/2 \rfloor}
	\end{align*}
	Since the information theory lower bound $\log_2(2^{\lfloor n/2 \rfloor}) = \left\lfloor \frac{n}{2} \right\rfloor \in \Omega(n)$, 
	$\Omega(n)$ is the lower bound of the worst case complexity.

\end{enumerate}

\end{document}

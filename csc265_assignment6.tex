\documentclass[10pt]{article}
\usepackage{amsmath}
\usepackage{amssymb}
\usepackage{amsthm}
\usepackage{fullpage}
\usepackage{comment}
\usepackage{graphicx}
\usepackage{listings}
\usepackage{enumitem}
\usepackage{scrextend}

\newcommand{\iimplies}{\mbox{ IMPLIES }}
\newcommand{\oor}{\mbox{ OR }}
\newcommand{\aand}{\mbox{ AND }}
\newcommand{\nnot}{\mbox{ NOT }}
\newcommand{\iiff}{\mbox{ IFF }}
\newcommand{\xxor}{\mbox{ XOR }}
%\newcommand{\algorithmicbreak}{\textbf{break}}
%\newcommand{\Break}{\State \algorithmicbreak}

\newtheorem{theorem}{Theorem}
\newtheorem{corollary}{Corollary}[theorem]
\newtheorem{lemma}[theorem]{Lemma}

\begin{document}

\begin{center}
{\bf \Large \bf CSC265H Assignment 6}
\end{center}

\noindent
Yibin Zhao\\
1002996261\\
NO OUTSIDE DISCUSSION\\
NO EXTRA MATERIAL CONSULTED\\

\begin{comment}
The median of a set of numbers is the element with rank $\lfloor (n+1)/2 \rfloor$.
\end{comment}

\begin{enumerate}
	
	\begin{comment}
	Prove that any decesion tree using 2-way comparisons that finds the median of $n$ distinct numbers 
	also determines the set of numbers greater than the median and the set of numbers smaller than the median.
	\end{comment}

	\item
	Since all elements are distinct, 2-way = comparison returns T iff the indices agree, and there is no point of camparing an element with itself in this problem. 
	Thus we can assume that there is no 2-way = comparison in this decision tree.

	\begin{lemma}
	If the median of an array $A$ of size $n > 1$ is $A[k]$, where $k \in \{1, 2, \cdots, n\}$, then $A[k]$ is at least compared once in the decision tree.
	\end{lemma}

	\begin{proof}
	Let $C$ be an array of size $n$ such that $C[k] > C[i] = A[i]. \forall i \neq k$.
	Then, the median of $C$ is not $C[k]$, in which case, the decision tree gives different output for $A$ and $C$.
	Suppose there is no comparison between the element on index $k$ and any other elements.
	Then, the decision tree shows the same output for $A$ and $C$ since all other elements of $A$ and $C$ are the same.
	This is a contradiction.

	Therefore, the lemma is true.
	\end{proof}

	\begin{lemma}
	
	\end{lemma}
	

	\begin{comment}
	Prove and $\Omega(n)$ lower bound on the worst case complexity of finding the median of $n$ numbers 
	using the decision tree with 2-way comparisons.
	\end{comment}

	\item By 1, we can make the output of the decision tree be the index of the median and the set of indecies of elements greater than the median and the set of indecies for smaller ones. \\
	Then, the total number of possible output for an array of size $n$ is
	\begin{align*}
		& n \cdot 
		\begin{pmatrix}
			n-1 \\
			\lfloor (n+1)/2 \rfloor - 1
		\end{pmatrix} \\
		=& \frac{n^{(\lfloor (n+1)/2 \rfloor)}}{\lfloor (n-1)/2 \rfloor!} \\
		\geq& \frac{n^{(\lfloor n/2 \rfloor)}}{\lfloor n/2 \rfloor!} \\
		=& \prod_{i=1}^{\lfloor n/2 \rfloor}\frac{n-i+1}{\lfloor n/2 \rfloor -i+1} \\
		\geq& 2^{\lfloor n/2 \rfloor}
	\end{align*}
	Since the information theory lower bound $\log_2(2^{\lfloor n/2 \rfloor}) = \left\lfloor \frac{n}{2} \right\rfloor \in \Omega(n)$, 
	$\Omega(n)$ is the lower bound of the worst case complexity.

\end{enumerate}

\end{document}

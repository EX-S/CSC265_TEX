\documentclass[10pt]{article}
\usepackage{amsmath}
\usepackage{amssymb}
\usepackage{amsthm}
\usepackage{fullpage}
\usepackage{comment}
\usepackage{graphicx}
\usepackage{listings}
\usepackage{enumitem}
\usepackage{scrextend}

\newcommand{\iimplies}{\mbox{ IMPLIES }}
\newcommand{\oor}{\mbox{ OR }}
\newcommand{\aand}{\mbox{ AND }}
\newcommand{\nnot}{\mbox{ NOT }}
\newcommand{\iiff}{\mbox{ IFF }}
\newcommand{\xxor}{\mbox{ XOR }}
%\newcommand{\algorithmicbreak}{\textbf{break}}
%\newcommand{\Break}{\State \algorithmicbreak}

\newtheorem{theorem}{Theorem}
\newtheorem{corollary}{Corollary}[theorem]
\newtheorem{lemma}[theorem]{Lemma}

\begin{document}

\begin{center}
{\bf \Large \bf CSC265H Assignment 6}
\end{center}

\noindent
Yibin Zhao\\
1002996261\\
Dicussed with Jiayi Zhang\\
NO EXTRA MATERIAL CONSULTED\\

\begin{comment}
The median of a set of numbers is the element with rank $\lfloor (n+1)/2 \rfloor$.
\end{comment}

\begin{enumerate}
	
	\begin{comment}
	Prove that any decesion tree using 2-way comparisons that finds the median of $n$ distinct numbers 
	also determines the set of numbers greater than the median and the set of numbers smaller than the median.
	\end{comment}

	\item

	I will prove this propesition by proving the following stronger theorem:
	
	\begin{theorem}
	Any decision tree using 2-way comparison that finds that $k$th largest element in $n$ distinct numbers
	also determines the set of numbers greater than the $k$th largest element and
	the set of numbers smaller than the $k$th largest element, where $k$ is any integer from 1 to $n$.
	\end{theorem}

	\begin{lemma}
	For a decision tree, the result of a 2-way = comparison is consistent for any input of size $n$.
	\end{lemma}

	\begin{proof}
	Consider a comparison $A[i] = A[j]$ in the decision tree, where $i, j \in \{1, 2, \cdots, n\}$.

	If $i = j$, then the comparison always return T. 
	Otherwise, it always returns F regardless of the input, since all numbers are distinct.
	\end{proof}

	Thus, the result of a $=$ comparison does not impliy anything about the structure of the input.

	Therefore, we can safely assume all comparisons below are 2-way $<, >, \leq, \geq$ comparisons, 
	unless I explicitly state that it is a $=$ comparison.

	\begin{lemma}
	If $k > 1$, then there is at least one comparison between the $k$th largest number and the $k-1$ largest number.
	\end{lemma}
	
	\begin{proof}
	Let A be any array with distinct numbers of size $n$.
	Since $k>1$, $n\geqk>1$.
	Let $p, q \in \{1, 2, \ldots, n\}$, such that $A[p]$ has rank $k-1$ and $A[q]$ has rank $k$.

	Consider an arbitrary decision tree with 2-way comparison that gives the index of the $k$th largest
	number in this array.
	Assume that there is no comparison between $A[p]$ and $A[q]$ in the path for this input $A$.
	Then, let $B[i] = A[i]$ for all $i$ not equal to $p$ or $q$, and let $B[p] = A[q]$ and $B[q] = A[p]$.

	Consider a comparison between $B[i]$ and $B[j]$, where $i \neq j$.
	By the assumption above, $\{i, j\} \neq \{p, q\}$. \\
	If neither $i, j$ equal to $p$ or $q$, the result is equal to the same comparison between $A[i]$ and $A[j]$. \\
	If one of $i, j$ is equal to $p$ or $q$. 
	Without loss of generality, suppose $j$ is not equal to either $p$ or $q$. 
	Then, if $A[j]$ has a rank smaller than $k-1$, then $A[j] = B[j] < A[p] = B[q]$ and $B[j] < A[q] = B[p]$. 
	This implies that the comparison between $B[i]$ and $B[j]$ is consistent with the comparison between $A[i]$ and $A[j]$.
	Otherwise, $A[j]$ has a rank greater than $k$, then $A[j] = B[j] > A[p] = B[q]$ and $B[j] > A[p] = B[q]$. 
	Thus, its also consistent in this case.
	Similarly, we can obtain that the result of this comparison is also consistent for $i$ not equal to either $p$ or $q$. \\
	
	Thus, the output is consistent for $A$ and $B$.
	However, the output is not correct for $B$. 
	This is a contradiction.

	Therefore, there is at least one comparison between $A[p]$ and $A[q]$.
	By generalization, the lemma is true.
	\end{proof}

	\begin{corrolary}
	If $k < n$, then there is at least one comparison between the $k$th largest number and the $k+1$ largest number.
	\end{corrolary}

	\begin{proof}
	Use similar approach as the lemma, not stating here.
	\end{proof}

	


	

	\begin{comment}
	Prove and $\Omega(n)$ lower bound on the worst case complexity of finding the median of $n$ numbers 
	using the decision tree with 2-way comparisons.
	\end{comment}

	\item By 1, we can make the output of the decision tree be the index of the median and the set of indecies of elements greater than the median and the set of indecies for smaller ones. \\
	Then, the total number of possible output for an array of size $n$ is
	\begin{align*}
		& n \cdot 
		\begin{pmatrix}
			n-1 \\
			\lfloor (n+1)/2 \rfloor - 1
		\end{pmatrix} \\
		=& \frac{n^{(\lfloor (n+1)/2 \rfloor)}}{\lfloor (n-1)/2 \rfloor!} \\
		\geq& \frac{n^{(\lfloor n/2 \rfloor)}}{\lfloor n/2 \rfloor!} \\
		=& \prod_{i=1}^{\lfloor n/2 \rfloor}\frac{n-i+1}{\lfloor n/2 \rfloor -i+1} \\
		\geq& 2^{\lfloor n/2 \rfloor}
	\end{align*}
	Since the information theory lower bound $\log_2(2^{\lfloor n/2 \rfloor}) = \left\lfloor \frac{n}{2} \right\rfloor \in \Omega(n)$, 
	$\Omega(n)$ is the lower bound of the worst case complexity.

\end{enumerate}

\end{document}

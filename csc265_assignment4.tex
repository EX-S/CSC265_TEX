\documentclass[10pt]{article}
\usepackage{amsmath}
\usepackage{amssymb}
\usepackage{fullpage}
\usepackage{comment}
\usepackage{graphicx}
\usepackage{listings}
\usepackage{enumitem}
\usepackage{algorithm}
\usepackage{algpseudocode}
\usepackage{verbatim}
\usepackage{amsthm}
\usepackage{scrextend}

\newcommand{\iimplies}{\mbox{ IMPLIES }}
\newcommand{\oor}{\mbox{ OR }}
\newcommand{\aand}{\mbox{ AND }}
\newcommand{\nnot}{\mbox{ NOT }}
\newcommand{\iiff}{\mbox{ IFF }}
\newcommand{\xxor}{\mbox{ XOR }}
\newcommand{\algorithmicbreak}{\textbf{break}}
\newcommand{\Break}{\State \algorithmicbreak}


\begin{document}


\begin{center}
{\bf \Large \bf CSC265H Assignment 4} \\
\end{center}

\noindent
Yibin Zhao\\
1002996261\\
NO OUTSIDE DISCUSSION\\
NO EXTRA MATERIAL CONSULTED\\

\begin{comment}
Consider the following algorithm:
INCREMENT(C)
choose X $\in \{0, 1\}$ such that Prob[X=1] = $frac{1}{2^C}$
if X=1 then C $\gets$ C+1
Suppose that C is initially 0.
\end{comment}

\begin{enumerate}

\begin{comment}
1. For each $v \in \{0, 1, 2, 3, 4\}$, what is the probablity that C = v after
INCREMENT(C) has been called 4 times? 
\end{comment}
	\item
		Let $X_i$ be $X$ for the $i$th time INCREMENT is called and let $C_i$ be
		the $C$ after the $i$th time INCREMENT(C) is called. \\
		Since $C$ is initially 0, it must increment by 1 after the first time
		INCREMENT(C) is called. \\
		$P(C_4=0) = P(X_1=0)P(X_2=0 \mid C_1=0)P(X_3=0 \mid C_2=0)P(X_4=0 \mid
		C_3=0) = 0 \times \cdots = 0$

		$P(C_4=1) = P(X_1=0) \times \cdots + P(X_1=1)P(X_2=0 \mid C_1=1)P(X_3=0
		\mid C_2=1)P(X_4=0 \mid C_3=1) = 0 + 1 \times \frac{1}{2^3} =
		\frac{1}{8}$ 

		$P(C_4=2) = P(X_1=0) \times \cdots + P(X_1=1)(P(X_2=0 \mid
		C_1=1)(P(X_3=0 \mid C_2=1)P(X_4=1 \mid C_3=1) + P(X_3=1 \mid
		C_2=1)P(X_4=0 \mid C_3=2)) + P(X_2=1 \mid C_1=1)P(X_3=0 \mid
		C_2=2)P(X_4=0 \mid C_3=2)) = 0 + 1 \times (\frac{1}{2} \times
		(\frac{1}{2} \times \frac{1}{2} + \frac{1}{2} \times \frac{3}{4}) +
		\frac{1}{2} \times \frac{3}{4} \times \frac{3}{4}) = \frac{19}{32}$ 

		$P(C_4=3) = P(X_1=0) \times \cdots + P(X_1=1)(P(X_2=0 \mid
		C_1=1)P(X_3=1 \mid C_2=1)P(X_4=1 \mid C_3=2) + P(X_2=1 \mid
		C_1=1)(P(X_3=1 \mid C_2=2)P(X_4=0 \mid C_3=3) + P(X_3=0 \mid
		C_2=2)P(X_4=1 \mid C_3=2))) = 0 + 1 \times
		(\frac{1}{2} \times \frac{1}{2} \times \frac{1}{4} +
		\frac{1}{2}(\frac{1}{4} \times \frac{7}{8}) + \frac{3}{4} \times
		\frac{1}{4}) = \frac{17}{64}$

		$P(C_4=4) = P(X_1 = 1)P(X_2=1 \mid C_1=1)P(X_3=1 \mid C_2=2)P(X_4=1
		\mid C_3=3) = 1 \times \frac{1}{2} \times \frac{1}{4} \times
		\frac{1}{8} = \frac{1}{64}$
	

\begin{comment}
2. Write a recurrence that describes the probablity $P_{t, v}$ that C = v after
INCREMENT(C) has been called t times, Explain your answer.
\end{comment}
	\item
		The recurrence is:
		\[ P_{t,v} = 
		\begin{cases}
			1 & \quad \text{if } t = 0 \aand v = 0 \\
			0 & \quad \text{if } t < v \\
			P_{t-1, v-1} \cdot \frac{1}{2^{v-1}} + P_{t-1,v} \cdot
			\left(1-\frac{1}{2^v} \right) & \quad \text{if } 0 \leq v \leq t
		\end{cases}
		\]
		where, $t, v \in \mathbb{N}$. 

		Before the first time, which is the 0th time, C is always 0 initially.
		Therefore, $P_{0,0} = 1$. 

		Since each time INCREMENT(C) is called, C could only increment by at
		most 1, which means that $v$ will never be larger than $t$. Thus, if $t
		< v$, $P_{t, v} = 0$. 

		Consider $t,v \in \mathbb{N}$ such that $v \leq t$. Since C either
		increment by 1 or retain its value each round, $P_{t,v}$ will only
		depend on $P_{t-1,v-1}$ and $P_{t-1, v}$. Suppose $C=v-1$ after
		INCREMENT(C) has been called $t-1$ times, then $P(X=1 \mid C=v-1) =
		\frac{1}{2^{v-1}}$. Suppose $C=v$ after INCREMENT(C) has been called
		$t-1$ times, then $P(X=0 \mid C=v) = 1-\frac{1}{2^v}$. Therefore, the
		recurrence in that case is $P_{t,v} = P_{t-1,v-1} \cdot
		\frac{1}{2^{v-1}} + P_{t-1,v-1} \cdot \left(1-\frac{1}{2^{v}} \right)$.

\begin{comment}
3. Prove that if INCREMENT(C) is called t times, then the expected value of
$2^C$ is t+1
\end{comment}
	\item
		\begin{proof}
			Let $P(t)$ be the predicate that "If INCREMENT(C) is called t
			times,  $E(2^C) = t+1$", where $t \in \mathbb{N}$. We want to prove
			that $\forall t \in \mathbb{N}. P(t)$. 

			\textbf{Base Case:} $t=0$
			\begin{addmargin}[1em]{0em}
				$C$ is 0 initially. \\
				$E(2^C) = E(2^0) = E(1) = 1 = t+1$. \\
				Therefore, $P(0)$ holds.
			\end{addmargin}

			\textbf{Inductive Step:}
			\begin{addmargin}[1em]{0em}
				Assume $P(t)$ is true for some $t \in \mathbb{N}$. \\
				We want to prove $P(t+1)$. \\
				Let $C_t$ be the $C$ after INCREMENT(C) is called $t$ times. \\
				By $P(t)$, $E(2^{C_t}) = \sum_{v=0}^{t}P_{t, v}2^v = t+1$. \\
				Then, by the recurrence in part 2,
				\begin{align*}
					E(2^{C_{t+1}})
					& = \sum_{v=0}^{t+1}P_{t+1,v}2^v \\
					& = \sum_{v=0}^{t+1}(P_{t,v} \cdot \left(1- \frac{1}{2^v}
					\right) + P_{t, v-1} \cdot \frac{1}{2^{v-1}}) \cdot 2^v \\
					&= \sum_{v=0}^{t+1}P_{t,v} \cdot \left(1- frac{1}{2^v}
					\right) \cdot 2^v + \sum_{v=0}^{t+1}P_{t,v-1} \cdot
					\frac{1}{2^{v-1}} \cdot 2^v \\
					&= \sum_{v=0}^{t}P_{t,v} \cdot \left(1 - \frac{1}{2^v}
					\right) \cdot 2^v + \sum_{v=0}^{t}P_{t, v} \cdot 2 &&
					\text{since } P_{t, t+1}=0, P_{t, -1}=0 \\
					&= \sum_{v=0}^{t}P_{t,v} \cdot 2^v - \sum_{v=0}^{t}P_{t,v}
					\cdot 1 + 2 \sum_{v=0}^{t}P_{t, v} \\
					&= \sum_{v=0}^{t}P_{t,v}2^v + \sum_{v=0}^{t}P_{t,v} \\
					&= E(2^{C_t}) + 1 \\
					&= (t+1) + 1\\
				\end{align*}
				Therefore, $P(t+1)$ is true.
			\end{addmargin}

			By induction, $\forall t \in \mathbb{N}. P(t)$. \\
			Thus, if INCREMENT(C) is called t times, then the expected value of
			$2^C$ is t+1. 
		\end{proof}

\end{enumerate}

\end{document}

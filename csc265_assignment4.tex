\documentclass[10pt]{article}
\usepackage{amsmath}
\usepackage{amssymb}
\usepackage{fullpage}
\usepackage{comment}
\usepackage{graphicx}
\usepackage{listings}
\usepackage{enumitem}
\usepackage{algorithm}
\usepackage{algpseudocode}
\usepackage{verbatim}

\newcommand{\iimplies}{\mbox{ IMPLIES }}
\newcommand{\oor}{\mbox{ OR }}
\newcommand{\aand}{\mbox{ AND }}
\newcommand{\nnot}{\mbox{ NOT }}
\newcommand{\iiff}{\mbox{ IFF }}
\newcommand{\xxor}{\mbox{ XOR }}
\newcommand{\algorithmicbreak}{\textbf{break}}
\newcommand{\Break}{\State \algorithmicbreak}


\begin{document}


\begin{center}
{\bf \Larger \bf CSC265H Assignment 4} \\
\end{center}

\noindent
Yibin Zhao\\
1002996261\\
NO OUTSIDE DISCUSSION\\
NO EXTRA MATERIAL CONSULTED\\

\begin{comment}
Consider the following algorithm:
INCREMENT(C)
choose X $\in \{0, 1\}$ such that Prob[X=1] = $frac{1}{2^C}$
if X=1 then C $\gets$ C+1
Suppose that C is initially 0.
\end{comment}

\begin{enumerate}

\begin{comment}
1. For each $v \in \{0, 1, 2, 3, 4\}$, what is the probablity that C = v after
INCREMENT(C) has been called 4 times? 
\end{comment}
\item

\begin{comment}
2. Write a recurrence that describes the probablity $P_{t, v}$ that C = v after
INCREMENT(C) has been called t times, Explain your answer.
\end{comment}
\item

\begin{comment}
3. Prove that if INCREMENT(C) is called t times, then the expected value of
$2^C$ is t+1
\end{comment}
\item

\end{enumerate}

\end{document}

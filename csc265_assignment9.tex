\documentclass[10pt]{article}
\usepackage{amsmath}
\usepackage{amssymb}
\usepackage{amsthm}
\usepackage{fullpage}
\usepackage{comment}
\usepackage{graphicx}
\usepackage{listings}
\usepackage{enumitem}
\usepackage{scrextend}
\usepackage{algorithm}
\usepackage{algpseudocode}

\newcommand{\iimplies}{\mbox{ IMPLIES }}
\newcommand{\oor}{\mbox{ OR }}
\newcommand{\aand}{\mbox{ AND }}
\newcommand{\nnot}{\mbox{ NOT }}
\newcommand{\iiff}{\mbox{ IFF }}
\newcommand{\xxor}{\mbox{ XOR }}
%\newcommand{\algorithmicbreak}{\textbf{break}}
%\newcommand{\Break}{\State \algorithmicbreak}

\newtheorem{theorem}{Theorem}
\newtheorem{corollary}{Corollary}[theorem]
\newtheorem{lemma}[theorem]{Lemma}

\begin{document}

\begin{center}
{\bf \Large \bf CSC265H Assignment 9}
\end{center}

\noindent
Yibin Zhao\\
1002996261\\
NO OUTSIDE DISCUSSION\\
NO EXTRA MATERIAL CONSULTED\\

\begin{comment}
	A list of the vertices of a graph is the order they are visited during a
	breadth first search is called a breadth first search order of the graph.
	Similarly, a depth first search order of a graph is a list of the vertices
	of a graph in the order they are visited during a depth first search. 
	Note that the breadth first search order and depth first search order of a
	graph can depend on the order in which the children oeach node are
	enumerated.

	For any (directed or undirected) graph $G=(V,E)$, its squeare $G^2=(V,E^2)$
	is the graph with the same vertex set $V$ and with edge set consisting of
	all pairs of vertices that are distance one or two apart in $G$. 
	In other $G^2$ can be obtained from $G$ by adding new edges joining all
	pairs of vertices that are distance two apart in $G$.
\end{comment}

\begin{enumerate}
	\item
	\begin{comment}
		Prove that any breadth first search order of $G$ is also a breadth
		frist search order of $G^2$
	\end{comment}
		\begin{lemma}
			Let $T$ be the BFS tree of $G$ with root $r$.
			For all $u, v \in V. u \neq v$, if $u$ has a child $c$ in $T$ and
			$v$ has a child $c'$ in $T$, then in any BFS order of $G$ start
			with $r$, $c'$ is visited after $c$ iff $v$ is visited after $u$.
		\end{lemma}
		\begin{proof}
			Suppose $v$ is visited after $u$, then $c'$ is enqueued after $c$
			and thus, $c'$ is visited after $c$. 

			Suppose $c'$ is visited after $c$, then $c'$ is enqueued after $c$.
			Notice that $c$ and $c'$ got enqueued in the loop where $u$ and $v$
			are visited respectively. 
			Since $u \neq v$, $c$ and $c'$ are not enqueued in the same loop.
			By the assumption, the loop for $v$ is executed after the loop for
			$u$, and thus, $v$ is visited after $u$.

			Therefore, the lemma is true.
		\end{proof}


		\begin{proof}
			Suppose that for an arbitrary graph $G$, there exists a breadth first
			searh order of $G$ that is not a breadth first search order of
			$G^2$. Let $\sigma: v_1, v_2, \cdots, v_n$ be the order.

			Let $T$ be the BFS tree of $G$ with root $v_1$, and $T'$ be the
			BFS tree of $G^2$ with root $v_1$ as well.

			Let $p_i: v_1, v_2, \cdots, v_i$ be the prefix of this order with size
			$i$, where $0 \leq i \leq n$.
			Then, there exists such $0 < i < n$ that there exists a BFS order of $G^2$
			with prefix $p_i$ and there is no BFS order of $G^2$ with prefix
			$p_{i+1}$.
			Note that since $v_1$ is the starting point of a BFS, $v_1$
			is in $p_1$ of a BFS order of $G^2$, and thus, it is safe to
			assume $i > 0$.
			There must exists a parent of $v_{i+1}$ in $T'$ ( denote it by
			parent'($v_{i+1}$)), since $v_{i+1}$ is not the root.
			Similarly, parent($v_{i+1}$) also exists.

			Now, consider all the cases that $p_{i+1}$ is invalid given that
			$p_i$ is valid. 
			
			Suppose parent'($v_{i+1}$) is not in $p_i$.
			Since $p_{i+1}$ is a valid prefix for $G$, parent($v_{i+1}$) is in
			$p_{i}$.
			If parent($v_{i+1}$) $=$ parent'($v_{i+1}$), this is
			a contradiction.
			If parent($v_{i+1}$) $\neq$ parent'($v_{i+1}$), then
			parent'($v_{i+1}$) must be visited before parent($v_{i+1}$), or
			otherwise, since there is an edge from parent($v_{i+1}$) to
			$v_{i+1}$, parent($v_{i+1}$) would be equal to parent'($v_{i+1}$).
			This is also a contradiction.

			Consider parent'($v_{i+1}$) is in $p_i$.
			Recall that parent($v_{i+1}$) must also be a vertex in $p_i$. 
			If $v_{i}$ has a parent in $T'$, then parent'($v_i$) $\neq$
			parent'($v_{i+1}$), or otherwise, $p_{i+1}$ is a valid prefix for
			$G^2$.

			Suppose parent'($v_i$) is visited after parent'($v_{i+1}$), then
			$v_{i+1}$ must got enqueued before $v_{i}$.
			This implies that $p_i$ is not a valid prefix, which is
			a contradiction.

			Consider parent'($v_i$) is visited before parent'($v_{i+1}$). 
			In order for $p_{i+1}$ be invalid given that $p_i$ is valid, there
			must exists a vertex $v_j$ in $p_i$ such that $v_j$ is in between
			parent'($v_i$) and parent'($v_{i+1}$) and $v_j$ has a child in $T'$.
			Denote the child of $v_j$ in $T'$ be $c'$. 
			Then, $c'$ is not a vertex in $p_{i+1}$. \\
			If parent($c'$) $=$ parent'($c'$), then parent($c'$) is visited
			before parent($v_{i+1}$), and thus, for the BFS order of $G$, $c'$
			must be visited before $v_{i+1}$.
			That is, $c'$ must be a vertex in $p_i$, which is contradict to the
			assumption. \\
			Otherwise, parent($c'$) $\neq$ parent'($c'$). 
			We can get parent'($c'$) is visited before parent($c$) is visited. 
			Then, there exists a vertex $u$ such that (parent'($c'$), $u$)
			and ($u$, $c'$) are edges in $E$. 
			Thus, $u$ must be visited before $v_{i+1}$, since
			parent'($c'$) is visited before parent($v_{i+1}$). 
			Also, either $u$ is parent($c'$) or $u$ is visited after
			parent($c'$), since otherwise, by the edge $(u, c')$, $c'$ must got
			enqueued before parent($c'$) is visited and that, $p_i$ is not
			a valid for graph $G$. 
			This implies that parent($c'$) is visited before parent($v_{i+1}$). 
			Then, $c'$ must also be visited before $v_{i+1}$ in the BFS order
			of $G$. 
			This is contradict to the assumption. 

			By proof of contradiction, there is no such $i$ that $p_i$ is valid
			and $p_{i+1}$ is invalid prefix of BFS order of $G^2$. 
			Recall that $p_1$ is always valid.
			By well ordering principle, $\forall i \in \{1, \cdots, n\}. p_i$
			is a valid prefix of BFS order of $G^2$.
			Notice that $p_n$ is just the order itself. 
			This directly follows that the order $v_1, \cdots, v_n$ is a BFS
			order of $G^2$. 

			By generalization, we can conclude that any BFS order of $G$ is
			a BFS order of $G^2$.


		\end{proof}

	\item
	\begin{comment}
		Prove that, for some graph $G$, there is a breadth first search order
		of $G^2$ that is not a breadth first search order of $G$
	\end{comment}

	\begin{proof}
		Let $V = \{v_1, v_2, v_3\}$ and $E = \{(v_1, v_2), (v_2, v_3)\}$.
		Let $G=(V,E)$ and $G^2 = (V, E^2)$.
		Notice that $(v_1, v_3) \in E^2$.
		Then, consider a BFS order $v_1, v_3, v_2$ of $G^2$.
		This is not a BFS order of $G$, since $v_2$ must be visited directly
		after $v_1$ if starting from $v_1$.

		Thus, the statement is true.
	\end{proof}

	\item
	\begin{comment}
		Prove that, for some graph $G$, there is a depth first search order of
		$G$ that is not a depth first search order of $G^2$
	\end{comment}
	\begin{proof}
		Let $V = \{v_1, v_2, v_3, v_4, v_5\}$ and $E = \{(v_1, v_2), (v_2,
		v_3), (v_3, v_1), (v_1, v_4), (v_2, v_5), (v_5, v_3)\}$.
		Then, $v_1, v_2, v_3, v_5, v_2$ is a DFS order of $G$.
		Notice that $(v_3, v_4) \in E^2$ since $(v_3, v_1)$ and $(v_1, v_4)$
		are edges in $E$.
		Consider performing the same DFS order on $G^2$.
		At $v_3$, there is an edge from $v_3$ to $v_2$ and there are no edges
		from $v_3$ to $v_5$.
		Thus, $v_2$ must be the next vertex, which is contradicted to the DFS
		order.
		Therefore, $v_1, v_2, v_3, v_5$, is not a DFS order of $G^2$.

		Hence, the statement is true.
	\end{proof}

	\item
	\begin{comment}
		Prove that, for some graph $G$, there is a depth frist search order of
		$G^2$ that is not a depth first search order of $G$
	\end{comment}
	\begin{proof}
		Let $V = \{v_1, v_2, v_3\}$ and $E = \{(v_1, v_2), (v_2, v_3)\}$.
		Let $G=(V,E)$ and $G^2 = (V, E^2)$.
		Then, $(v_1, v_3) \in E^2$.
		Consider a DFS order $v_1, v_3, v_2$ of $G^2$.
		This is not a DFS order of $G$, $v_2$ must be visited directly after $v_i$.

		Thus, the statement is true.
	\end{proof}

\end{enumerate}

\end{document}

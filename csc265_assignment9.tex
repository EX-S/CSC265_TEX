\documentclass[10pt]{article}
\usepackage{amsmath}
\usepackage{amssymb}
\usepackage{amsthm}
\usepackage{fullpage}
\usepackage{comment}
\usepackage{graphicx}
\usepackage{listings}
\usepackage{enumitem}
\usepackage{scrextend}
\usepackage{algorithm}
\usepackage{algpseudocode}

\newcommand{\iimplies}{\mbox{ IMPLIES }}
\newcommand{\oor}{\mbox{ OR }}
\newcommand{\aand}{\mbox{ AND }}
\newcommand{\nnot}{\mbox{ NOT }}
\newcommand{\iiff}{\mbox{ IFF }}
\newcommand{\xxor}{\mbox{ XOR }}
%\newcommand{\algorithmicbreak}{\textbf{break}}
%\newcommand{\Break}{\State \algorithmicbreak}

\newtheorem{theorem}{Theorem}
\newtheorem{corollary}{Corollary}[theorem]
\newtheorem{lemma}[theorem]{Lemma}

\begin{document}

\begin{center}
{\bf \Large \bf CSC265H Assignment 9}
\end{center}

\noindent
Yibin Zhao\\
1002996261\\
NO OUTSIDE DISCUSSION\\
NO EXTRA MATERIAL CONSULTED\\

\begin{comment}
	A list of the vertices of a graph is the order they are visited during a
	breadth first search is called a breadth first search order of the graph.
	Similarly, a depth first search order of a graph is a list of the vertices
	of a graph in the order they are visited during a depth first search. 
	Note that the breadth first search order and depth first search order of a
	graph can depend on the order in which the children oeach node are
	enumerated.

	For any (directed or undirected) graph $G=(V,E)$, its squeare $G^2=(V,E^2)$
	is the graph with the same vertex set $V$ and with edge set consisting of
	all pairs of vertices that are distance one or two apart in $G$. 
	In other $G^2$ can be obtained from $G$ by adding new edges joining all
	pairs of vertices that are distance two apart in $G$.
\end{comment}

\begin{enumerate}
	\item
	\begin{comment}
		Prove that any breadth first search order of $G$ is also a breadth
		frist search order of $G^2$
	\end{comment}
		\begin{lemma}
			Let $T$ be the BFS tree of $G$ with root $r$.
			For all $u, v \in V. u \neq v$, if $u$ has a child $c$ in $T$ and
			$v$ has a child $c'$ in $T$, then in any BFS order of $G$ start
			with $r$, $c'$ is visited after $c$ iff $v$ is visited after $u$.
		\end{lemma}
		\begin{proof}
			Suppose $v$ is visited after $u$, then $c'$ is enqueued after $c$
			and thus, $c'$ is visited after $c$. 

			Suppose $c'$ is visited after $c$, then $c'$ is enqueued after $c$.
			Notice that $c$ and $c'$ got enqueued in the loop where $u$ and $v$
			are visited respectively. 
			Since $u \neq v$, $c$ and $c'$ are not enqueued in the same loop.
			By the assumption, the loop for $v$ is executed after the loop for
			$u$, and thus, $v$ is visited after $u$.

			Therefore, the lemma is true.
		\end{proof}

		\begin{lemma}
			If $v$ appears after $u$ in a BFS order of $G$ starting from $w$,
			then the height of $v$ is greater than or equal to the height of
			$u$ in the BFS tree with root $w$.
		\end{lemma}
		\begin{proof}
			Perform induction on the height of $u$ to prove this.

			Let $P(h):$ "For all $u$ with height $h$, for all $v$ appears after
			$u$ in a BFS order, the height of $v$ is greater than or equal to
			the height of $u$".
			We want to prove that $\forall h \in \mathbb{N}. P(h)$. 

			For simplicity, let $h(v)$ denote the the height of any vertex $v$
			in the BFS tree. 

			\textbf{Base Case:} $h=0$
			\begin{addmargin}[1em]{0em}
				Then $u$ can only be $w$.
				Since $w$ is just the root, $h(w) = 0$ and $h(v) \geq 0$ for
				any vertex $v$.
				Thus, $P(0)$ is true.
			\end{addmargin}

			Suppose $P(x)$ is true for $x = h$, where $h \in \mathbb{N}$
			is arbitrary.
			Let $u$ be any node s.t. $h(u) =  h+1$.
			Then, $u$ is not the root.
			Thus, parent($u$) exists, and it appears before $u$ in any BFS
			order with root $w$.
			The height of parent($u$) is exactly $h$.
			Consider a $v$ which appears after $u$ in a BFS order.
			$v$ cannot be the root, or otherwise the root appears twice in the
			order. 
			It follows that $h(v) \geq 1$.
			Then parent($v$) exists.
			If parent($u$) $=$ parent($v$), then $h(v) = h+1$ and we are done.
			Otherwise, since $v$ appears after $u$, by BFS, parent($v$) must
			also appears after parent($u$).
			By the assuption above, $h = h(\text{parent}(u)) \leq
			h(\text{parent}(v))$.
			Then, $h(v) > h(\text{parent}(v)) \geq h$, and thus, $h(v) \geq
			h+1$. 
			Therefore, $P(h+1)$ is true.

			By induction, $\forall h \in \mathbb{N}. P(h)$.
			
			Hence, the lemma is true.
		\end{proof}

		\begin{proof}
			Suppose that for an arbitrary graph $G$, there exists a breadth first
			searh order of $G$ that is not a breadth first search order of
			$G^2$. Let $\sigma: v_1, v_2, \cdots, v_n$ be the order.

			Let $T$ be the BFS tree of $G$ with root $v_1$, and $T'$ be the
			BFS tree of $G^2$ with root $v_1$ as well.
			Let $h(v)$ denote the height of $v$ in $T$ and $h'(v)$ bw the
			height of $v$ in $T'$.


		\end{proof}

	\item
	\begin{comment}
		Prove that, for some graph $G$, there is a breadth first search order
		of $G^2$ that is not a breadth first search order of $G$
	\end{comment}

	\begin{proof}
		Let $V = \{v_1, v_2, v_3\}$ and $E = \{(v_1, v_2), (v_2, v_3)\}$.
		Let $G=(V,E)$ and $G^2 = (V< E^2)$.
		Notice that $(v_1, v_3) \in E^2$.
		Then, consider a BFS order $v_1, v_3, v_2$ of $G^2$.
		This is not a BFS order of $G$, since $v_2$ must be visited directly
		after $v_1$ if starting from $v_1$.

		Thus, the statement is true.
	\end{proof}

	\item
	\begin{comment}
		Prove that, for some graph $G$, there is a depth first search order of
		$G$ that is not a depth first search order of $G^2$
	\end{comment}
	\begin{proof}
		Let $V = \{v_1, v_2, v_3, v_4, v_5\}$ and $E = \{(v_1, v_2), (v_2,
		v_3), (v_3, v_1), (v_1, v_4), (v_2, v_5), (v_5, v_3)\}$.
		Then, $v_1, v_2, v_3, v_5, v_2$ is a DFS order of $G$.
		Notice that $(v_3, v_4) \in E^2$ since $(v_3, v_1)$ and $(v_1, v_2)$
		are edges in $E$.
		Consider performing the same DFS order on $G^2$.
		At $v_3$, there is an edge from $v_3$ to $v_2$ and there is no edge
		from $v_3$ to $v_5$.
		Thus, $v_2$ must be next vertex, which is contradicted to the DFS
		order.
		Therefore, $v_1, v_2, v_3, v_5$, is not a DFS order of $G^2$.

		Hence, the statement if true.
	\end{proof}

	\item
	\begin{comment}
		Prove that, for some graph $G$, there is a depth frist search order of
		$G^2$ that is not a depth first search order of $G$
	\end{comment}
	\begin{proof}
		Let $V = \{v_1, v_2, v_3\}$ and $E = \{(v_1, v_2), (v_2, v_3)\}$.
		Let $G=(V,E)$ and $G^2 = (V, E^2)$.
		Then, $(v_1, v_3) \in E^2$.
		Consider a DFS order $v_1, v_3, v_2$ of $G^2$.
		This is not a DFS order of $G$, since starting from $v_1$, $v_2$ must
		be visited directly afterwords.

		Thus, the statement is true.
	\end{proof}

\end{enumerate}

\end{document}

\documentclass[10pt]{article}
\usepackage{amsmath}
\usepackage{amssymb}
\usepackage{fullpage}
\usepackage{comment}
\usepackage{graphicx}
\usepackage{listings}
\usepackage[shortlabels]{enumitem}
%\usepackage{scrextend}
\usepackage{algorithm}
\usepackage{algpseudocode}
\usepackage{verbatim}

\newcommand{\iimplies}{\mbox{ IMPLIES }}
\newcommand{\oor}{\mbox{ OR }}
\newcommand{\aand}{\mbox{ AND }}
\newcommand{\nnot}{\mbox{ NOT }}
\newcommand{\iiff}{\mbox{ IFF }}
\newcommand{\xxor}{\mbox{ XOR }}
\newcommand{\algorithmicbreak}{\textbf{break}}
\newcommand{\Break}{\State \algorithmicbreak}

\begin{document}

\begin{center}
{\bf \Large \bf CSC265H Assignment 3}\\
\end{center}

\noindent
Yibin Zhao\\
1002996261\\
NO OUTSIDE DISCUSSION\\
NO EXTRA MATERIAL CONSULTED\\

\begin{comment}
Consider a sequence A[1..n], each of whose entries is an element of $\{1,
\ldots, n^2\}$, that supports two operations:
	UPDATE(i, v), for $1 \leq i \leq n$ and $v \in \{1, \ldots, n^2\}$, which
	sets A[i] to v and 
	ATMOST2(i,j), for $1 \leq i < j \leq n$, which returns T if A[i..j]
	contains at most two different numbers and returns F is A[i..j] contains at
	least three different numbers.

1. Construct a data structure that performs both opertations in O(log n) time.

2. Construct a data structure that performs ATMOST2 in O(1) time and UPDATE in
O(n) time.

Both data stuctures must use O(n) words, each with O(log n bits.

For each data structure:
	a. Brirfly describe it and explain why it has the required space
	complexity.
	b. Draw a picture of it when A[1..8]= [2,3,2,2,2,1,3,1].
	c. Describe how to perform UPDATE.
	d. Explain why the UPDATE algorithm is correct.
	e. Explain ehy the UPDATE algorithm had the required time complexity.
	f. Describe how to perform ATMOST2.
	g. Explain why the ATMOST2 algorithm is correct.
	h. Explain why the ATMOST2 algorithm has the required time complexity.
\end{comment}

\subsection*{2. Construct a data structure that performs ATMOST2 in O(1) tim
and UPDATE in O(n) time.}

\begin{enumerate}[a.]
	\item % a
		My data structure for this part is a data stucture with three arrays
		A[1..n], AT2[1..n] and E[1..n]. 
		A[1..n] is the array equivalent to the sequence discribed in the
		problem. 
		For any $i$ in $1, \ldots, n-1$, AT2[i] stores the value of the largest
		$j$ such that ATMOST(i,j) should returns T, and E[i] stores the number
		different from A[i] within A[i..AT2[i]]. 
		For simplicity, AT2[n] $= n$, and E[i]=0 if there is no such different
		number. \\

		For each $v \in \{1, \ldots, n^2\}$, it needs $\lceil \log_2(n^2)
		\rceil = \lceil 2\log_2(n) \rceil$ bits, which is in $O(log n)$. 
		There are 3 arrays, each with $n$ words, and in total there are $3n$,
		which is in $O(n)$. 

	\item % b
		$$
		\begin{matrix}
				& 1 & 2 & 3 & 4 & 5 & 6 & 7 & 8 \\
			\hline
			A	& 2 & 3 & 2 & 2 & 2 & 1 & 3 & 1 \\
			AT2	& 5 & 5 & 6 & 6 & 6 & 8 & 8 & 8 \\
			E	& 3 & 2 & 1 & 1 & 1 & 3 & 1 & 0
		\end{matrix}
		$$
	\item % c. Describe how to perform UPDATE
		\begin{algorithmic}[1]
			\Function{UPDATE}{$i,v$}
				\State $j \gets i$
				\State $A[i] \gets v$
				\State $k \gets i$
				\While{$j > 0 \aand k \geq i-1$}
					\If{$j < n$}
						\If{A[j] = A[j+1] $\oor$ A[j] = E[j+1]}
							\State AT2[j] $\gets$ AT2[j+1]
							\State E[j] $\gets$ A[j+1]+E[j+1]-A[j]
						\Else
							\State E[j] $\gets$ A[j+1]
							\State $k \gets j+1$
							\While{$k \leq n$}
								\If{A[k] $\neq$ A[j] $\aand$ A[k] $\neq$ E[j]}
									\Break
								\EndIf
								\State $k \gets k+1$
							\EndWhile
							\State $k \gets k-1$
							\State AT2[j] $\gets k$
						\EndIf
					\EndIf
					\State $j \gets j-1$
				\EndWhile
			\EndFunction
		\end{algorithmic}
	\item % d. Explain why the UPDATE algorithm is correct.
		Suppose we perform the update algorithm to a sequence A[1..n]. 
		Since for any $p$, AT2[p] and E[p] only depend on the squence A[p..n],
		suppose we do UPDATA(i, v), we don't need to change anything for the
		elements behind $i$. 
		However, we might need to modify AT2 and E for all elements before $i$.
		That is why I assign $i$ to $j$ before the outer while loop and have $j>0$
		as one of the condition for the outer while loop. 

		It is easy to see that ATMOST2 will always return true for any two
		consecutive indices.
		Therefore, AT2 must always exists for any index from 1 to n-1. 
		And the value of AT2 and E at index $n$, by definition, will never change. 
		Thus, we can ignore the case when $j = n$.

		Consider two consecutive elements with the same value, then they have
		the same AT2 and E value.
		(Since it is an explanation, I will not prove this trivial claim) 
		Also, if two consecutive elements on $j, j+1$ satisfy E[j+1] =
		A[j], then E[j] = A[j+1] and AT2[j] = AT2[j+1]. 
		(The following is a pseudo proof for this "non-trivial" claim)
		By definition of E, E[j+1] $\neq$ A[j+1], and by the condition E[j+1] =
		A[j]. 
		Also ATMOST(j, j+1) always returns true. 
		Therefore, E[j] = A[j+1].
		Since the sequence A[j+1..AT2[j+1]] only has two distinct values A[j+1]
		and E[j+1], and E[j+1] = A[j], A[j..AT2[j+1]] will also have the same
		two distinct values.
		Then AT2[j] $\leq$ AT2[j+1].
		By definition of AT2, AT2[j] cannot exceeds AT2[j+1], and thus, AT2[j]
		= AT2[j+1]. \\
		By these two claims above, the if statement from line 7 to 9 is
		correct.

		Now, consider the case when none of the conditions in either of those two claims is
		satisfied. 
		E[j] will still equal to A[j+1] by the fact that ATMOST2(j,j+1) always
		returns T.
		Consider the inner while loop. 
		If A[k] $\neq$ A[j] and A[k] $\neq$ E[j], then the sequence A[j..k] has
		three distinct values A[j], E[j] and A[k]. 
		If follows that ATMOST2(j, k) returns F.
		Since $k$ goes from smaller to larger, after the while loop, $k$ will
		be the smallest value such that ATMOST2(j, k) returns F, or $k = n+1$
		by the condition of the loop. 
		Then, after line 17, $k$ will be the largest value such that ATMOST2(j,
		k) returns T.
		That is, AT2[j] should equal to $k$.

		Consider $k < i-1$ after some iteration of the while loop. 
		Then, the else block form line 10 to 21 must be executed.
		Since $j$ is subtracted by 1 on line 23, AT2[j+1] = $k$.
		Then for all $p \leq j$, AT2[p] $\leq$ AT2[j+1] $< i-1$. 
		The sequence A[1..i-1] are not being modified, which means that for all
		$p \leq j$, the subsequence A[p..A[p]+1] remain the same.
		Thus, for all $p \leq j$, AT2[p] and E[p] should keep the same value. 
		Therefore, we can terminate the algorithm, since nothing in the data
		structure needs to be updated.

	\item % e. Explain why the UPDATE algorithm has the required time complixity.
		Line 2-4 are all assignments and runs in constant time.

		Since $j = i$ before the first iteration and each time $j$ reduces by 1, the outer
		while loop iterates at most $i$ times. 
		The if statement from line 7 to 9 runs in constant time.
		In the else block from line 11 to 20, line 11, 12, 19 and 20 are
		assignments and they run in constant time. 

		The inner while loop runs in $O(n)$, since line 17 runs in constant
		time and the loop runs at most $n$ times by the condition.
		In order to prove this algorithm runs in O(n) time, we will need to
		prove that the inner while loop is executed by at most a constant
		time, and in exact, 3.

		We know that for any $j, j+1$ in 1...n, AT2[j] $\leq$ AT2[j+1], and by
		induction, for any $p < q$ in 1...n, AT2[p] $\leq$ AT2[q].
		If AT2[j] = AT2[j+1], then either A[j] = A[j+1] or E[j] = A[j+1], in
		which case, the if statement on line 6 to 8 is executed. 
		Therefore, once the inner while loop is executed, AT2[j] will always be
		smaller than AT2[j+1] after line 20 is executed.
		Consider the iteration of the outter while loop where the inner while
		loop is executed for the second time. 
		Then, after line 20, AT2[j] $\leq i-1$, or otherwise, A[j+1]...A[i] all have
		the same value and the inner while loop has not been executed before
		this iteration.
		Let $j_0 = j$.
		Now, suppose the inner while loop is executed for the third time in some
		iteration of the outer while loop.
		Since $j < j_0$, after line 20 is performed, AT2[j] < AT2[$j_0$].
		Thus, AT2[j] = k < $i-1$, and the outer while loop halts after this
		iteration.
		Therefore, the inner while loop (or the else block) at most executes 3 times. 
		
		Since the else block only runs for a constant times, and each time the
		block runs in $O(n)$, it takes $O(n)$ time for the whole algorithm. 
		The outer while loop runs at most n times. 
		Exclude the times where else block is performed, then the outer while
		loop is in $O(n)$ time since the if statement runs in constant time.
		Combine these two and line 2-4, the total run time for this algorithm
		is $O(n)$. 

	\item % f
		\begin{algorithmic}[1]
			\Function{ATMOST}{$i, j$}
				\If{AT2[i] $< j$}
					\State \Return F
				\Else
					\State \Return T
				\EndIf
			\EndFunction
		\end{algorithmic}

	\item % g
		Suppose we perform ATMOST on A[1..n] with arguments $i, j$. By the
		definition of AT2, if $j >$ AT2[i], then there are at least three
		different numbers in A[i..j], in which case, the algorithm should
		return F, and it returns F by the implimentation above. Also, by
		definition, there are at most two different numbers in A[i..AT2[i]]. If
		$j \leq$ AT2, A[i..j] is within A[i..AT2[i]], and thus, ATMOST2(i, j)
		should return F. By the implimentation, it also returns F. Therefore,
		the algorithm is correct. 

	\item % h
		The algorithm only needs one comparison and one return step. Both of
		them runs in $O(1)$. Thus, the algorithm runs in $O(1)$ time.

\end{enumerate}		
		
\end{document}

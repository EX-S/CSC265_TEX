\documentclass[10pt]{article}
\usepackage{amsmath}
\usepackage{amssymb}
\usepackage{amsthm}
\usepackage{fullpage}
\usepackage{comment}
\usepackage{graphicx}
\usepackage{listings}
\usepackage{enumitem}
\usepackage{scrextend}
\usepackage{algorithm}
\usepackage{algpseudocode}
\usepackage[margin=.5in]{geometry}
\usepackage{parcolumns}
\usepackage{kantlipsum}

\newcommand{\iimplies}{\mbox{ IMPLIES }}
\newcommand{\oor}{\mbox{ OR }}
\newcommand{\aand}{\mbox{ AND }}
\newcommand{\nnot}{\mbox{ NOT }}
\newcommand{\iiff}{\mbox{ IFF }}
\newcommand{\xxor}{\mbox{ XOR }}
%\newcommand{\algorithmicbreak}{\textbf{break}}
%\newcommand{\Break}{\State \algorithmicbreak}

\newtheorem{theorem}{Theorem}
\newtheorem{corollary}{Corollary}[theorem]
\newtheorem{lemma}[theorem]{Lemma}

\begin{document}

\begin{center}
{\bf \Large \bf CSC265H Assignment 10}
\end{center}

\noindent
Yibin Zhao\\
1002996261\\
NO OUTSIDE DISCUSSION\\
NO EXTRA MATERIAL CONSULTED\\

\begin{comment}
	Give an algorithm that, given a directed graph $G=(V,E)$ which contains no
	directed cycle and a weight function $w: E \rightarrow \mathbb{R}$, find
	a maximum weight path in $O(|V|+|E|)$ time.

	Prove that your algorithm is correct and has the required time complexity.
\end{comment}

\begin{parcolumns}{2}

\colchunk{
\begin{algorithmic}[1]
	\Function{MAXPATH}{$G$}
		\ForAll{$v \in V$}
			\State $d[v] \gets -1$
			\Comment{$d$ is global}
		\EndFor
		\State $V' \gets \phi$
		\State max $\gets -1$
		\ForAll{$v \in V$}
			\If{$d[v] < 0$}
				\State \Call{MAXSPATH}{$G, v$}
			\EndIf
			\If{$d[v] >$ max}
				\State max $\gets d[v]$
			\EndIf
			\State $V' \gets V' \cup \{v\}$
		\EndFor
		\State \Return max
	\EndFunction
\end{algorithmic}
}

\colchunk{
\begin{algorithmic}[1]
	\Function{MAXSPATH}{$G, s$}
		\ForAll{$u \in V$ such that $(s, u) \in E$}
			\If{$d[u] < 0$}
				\State \Call{MAXSPATH}{$G, u$}
			\EndIf
			\If{$d[s] < d[u] + w((s, u))$}
				\State $d[s] \gets d[u] + w((s, u))$ 
			\EndIf
		\EndFor
		\If{$d[s] < 0$}
			\State $d[s] \gets 0$
		\EndIf
	\EndFunction
\end{algorithmic}
}
\end{parcolumns}

\vspace{5mm}

\setlength\parindent{0pt}
Proof of correctness and complexity:

\begin{lemma}
	For any graph $G$, if $G$ contains no directed cycle, then there exists $v
	\in V$ such that $v$ has no out-neighbour, and each path consists
	of at most $n$ vertices
\end{lemma}
\begin{proof}
	Let $|V| = n$.

	Suppose there exists a path $p$ with $n+1$ vertices.
	Since $n+1 > n$, there must be at least two vertices $p_i$ and $p_j$, $i
	\neq j$ and $p_i = p_j$.
	Then, there is a directed cycle in the graph.
	This is a contradiction.
	
	Therefore, each path consists of at most $n$ vertices. 
	
	Suppose that for each $v$ in $V$, $v$ has a out-neighbour.
	Then, since each vertex has a out-neighbout, there exists a path $p$(not
	necessarily simple) with $n+1$ vertices.
	This contradicts to the above conclusion. 

	Therefore, there exists some vertex with no out-neighbour.
\end{proof}

\begin{lemma}
	$d[s] \geq 0$ iff MAXSPATH($G, s$) has been executed.
\end{lemma}
\begin{proof}
	All $d[v]$ are initialized as -1.

	Suppose $d[s] \geq 0$, then, by the algorithm, it is easy to see that
	MAXPATH($G,s$) must have been executed.

	Suppose MAXPATH($G,s$) has been executed for some $s \in V$.
	After line 9, if $d[s]$ is still negative, then the if block on line 10-12
	is executed and $d[s] \geq 0$.
	Otherwise, $d[s] \geq 0$ alreay, and we are done.

	Therefore, the statement is true.
\end{proof}

\begin{lemma}
	For any vertex $s$, after MAXSPATH($G, s$) is performed, $d[s]$ is the
	weight of the maximum weight path with starting vertex $s$.
\end{lemma}
\begin{proof}
	By lemma1, there always exists some vertex $v \in V$ such that $v$ has no
	out-neighbours. 
	Thus, we can perform structual induction to prove this statement. 
	Let $P(s)$ be the predicate that "After MAXSPATH($G, s$) is performed,
	$d[s]$ is the weight of the maximum weight path with starting vertex $s$",
	where $s \in V$. 
	We want to prove $\forall s \in V. P(s)$

	\textbf{Base Case:} $s$ has no out-neighbours.
	\begin{addmargin}[1em]{0em}
		Then the maximum weight path with staring point $s$ is just 0. 
		
		Notice that, the for loop on line 2-9 is not executed.
		By line 7 in MAXPATH and line 3 in MAXSPATH, we guarantee to have
		$d[s] < 0$ before MAXSPATH($G, s$) is performed. 
		Then, the if block on line 10-12 must be executed, and thus $d[s] = 0$
		after MAXSPATH($G, s$) is executed. 

		Therefore, $P(s)$ is true.
	\end{addmargin}

	\textbf{Inductive Step:}
	\begin{addmargin}[1em]{0em}
		Suppose $s$ has a out-neighbour and for all out-neighbours $u$ of $s$,
		$P(u)$ holds. 
		Want to prove $P(s)$.

		By line 7 in MAXPATH and line 3 in MAXSPATH, we guarantee to have $d[s]
		< 0$ before MAXSPATH($G, s$) is performed.

		Suppose after the MAXSPATH is performed, $d[s]$ is not the maximum
		weight path with source $s$.
		Then, there exists some path $p: s, p_1, p_2, \ldots, p_k$ such that
		$p$ is the path with maximum weight from source $s$. 
		This follows that $d[s] < w(p)$ in any iteration on the for loop.
		Since, there is no directed cycles in $G$, $p$ is a simple path. 
		
		If $k \geq 1$
		Then, $(s, p_1) \in E$, and that that, $p_1$ is a out-neighbour of
		$s$. 
		Then, by the assumption, the weight of $p_1, p_2, \ldots, p_k$ is at
		most $d[p_1]$.
		Then, $w(p_1) \leq w((s, p_1)) + d[p_1]$.
		Consider during the iteration where $u = p_1$.
		At line 6, $d[s] < d[u] + w((s, u))$ is  satisfied and thus, line
		7 is performed.
		Therefore, after this iteration, $d[s] = d[u] + w((s, u)) \geq w(p_1)$.
		By the implementation, $d[s]$ is non-decreasing after each iteration of
		the for loop.
		Then, after the last iteration, $d[s] \geq d[p_1] + w((s, p_1)) \geq
		w(p_1)$.
		This is a contradiction. 

		Consider $k = 0$. 
		Then the maximum weight path is just $p: s$.
		(e.g. All out-edges from $s$ are negative-weighted) 
		If $d[s] \geq 0$ after line 9, then $d[s] \geq w(p) = 0$.
		Otherwise, line 11 is performed and $d[s] = 0 \geq w(p) = 0$.
		This is a contradiction. 

		Conbine these two cases, we can conclude that $d[s]$ is the weight of
		the maximum weight path with source $s$.
		That is, $P(s)$ is true.
	\end{addmargin}

	By induction, we can conclude that the statement is true. 
\end{proof}


Consider the loop invariant for the for loop on line 7-15 in MAXPATH.

\textbf{Loop Invariant:} 
	(1) $\forall v \in V'. d[v]$ is the weight of the maximum weight path from
	source $v$.\\ 
	(2) max $= \max\{d[v] | v \in V'\}$.

\begin{proof}
	Initially, $V'$ is empty, and thus, the loop invariant holds (1) automatically.
	Since all $d[v] = -1 = \max$, loop invariant (2) also holds. 

	Now, consider any iteration of the for loop.

	If $d[v] < 0$ at the beginning of this iteration, then the condition on
	line 8 is satisfied and MAXSPATH($G, v$) is executed. 
	By lemma 3, after line 9, $d[v]$ is the weight of the maximum weight path
	from $v$ and the loop invariant(1) holds.
	
	Otherwise, $d[v] \geq 0$. 
	By lemma 2, MAXSPATH must has been executed before this iteration.
	Then, by lemma 3, $d[v]$ is the weight of the maximum weight path from $v$
	and the loop invariant (1) hold.

	Combine these two cases, we can conclude that after each iteration of this
	for loop, loop invariant (1) holds. 

	To prove loop invariant (2), use induction.
	Base case is shown at the beginning of this proof. 

	Let $v_i$ be the vertex $v$ in the $i$th iteration.
	Suppose after the $i$th iteration, loop invariant (2) is true, where $i \in
	\mathbb{N}$.
	Consider the $i+1$th iteration. 

	If line 12 is performed, then the new max is greater than the old max and
	is equal to $d[v_{i+1}]$.
	Immediatly After this iteration, $\max\{d[v] | v \in V'\} = d[v_{i+1}]
	= \max$
	Thus, loop invariant (2) holds.

	Otherwise, $d[v_{i+1}] \leq max$, and by the assuption, loop invariant
	holds. 

	Therefore, by these two cases, loop invariant (2) holds for the $i+1$th
	iteration. 

	Hence, by induction, loop invariant (2) is true.
\end{proof}

Correctness of MAXSPATH($G,s$)

Precondition: $G=(V,E)$ is a directed graph with no directed cycles, $s \in V$,
$d[s] = -1$. \\
Postcondition: $G$ is unchanged and $d[s]$ is the weight of the maximum path
with source $s$.

\begin{proof}
	By lemma 3, partial correctness holds.

	By lemma 1, all path starting from $s$ consists of at most $n$ vertices. 
	Thus, if MAXSPATH$(G, s)$ is called, it guarantee to have at most $n$
	recursive depth. 

	The base case is when $s$ has no out-neighbour.
	The for loop is not performed and the program halts.

	Suppose for all out-neighbour $u$ of $s$, MAXSPATH($G,u$) halts.
	Then, the for loop runs at most $n$ times, and each iteration halts. 
	Thus, the algorithms halts for MAXSPATH($G,s$).

	Hence, by induction, we can conclude that the algorithms halts.

	Therefore, the algorithm is correct.
\end{proof}

Correctness of MAXPATH($G$)

Precondition: $G=(V,E)$ is a directed graph with no directed cycles. \\
Postcondition: $G$ is unchanged and the function returns the weight of the
maximum weight path.

\begin{proof}
	By line 7 and 14, after the loop on line 7-15 halts, $V' = V$.
	Then, by loop invariant, after the for loop on line 7-15 is executed, $\max
	= \max\{d[v] | v \in V\}$, and by the correctness of MAXSPATH, each
	$d[v]$ is the weight of the maximum path starting from $v$. 
	Note that the weight of a maximum weight path is always non-negative, since
	a path with no edges is of weight 0, and thus, we are safe to initialize
	each $d[v]$ and max by -1. 

	Let the weight of the maximum weight path $p: p_1, p_2, \cdots, p_k$ be
	$w(p)$, then after line 15, $d[p] = w(p)$ and $w(p) \geq d[v]$ for all $v
	\in V$. 
	Since $\max = \max\{d[v] | v \in V\}$, $\max = w(p)$, and thus, the
	algorithm returns the correct answer. 

	Consider the halting of this algorithm.

	The first for loop runs $|V| = n$ iterations and each iteration runs in
	constant time.
	
	Consider the second for loop from line 7-15. 
	There are in tital $n$ iterations, and for each iteration, by the
	correctness of MAXSPATH, the iteration halts.
	Thus the second for loop halts. 

	Therefore, this algorithm halts. 

	Hence, by the partial correctness and the halting of this algorithm, the
	algorithm is correct.
\end{proof}

Complexily of MAXPATH is $O(|E|+|V|)$
\begin{proof}
	Let the number of comparisons and assignments be the run time of this
	algorithm. 
	Let $C(A)$ denote the run time of algorithm $A$.
	
	By lemma 2 and conditions on line 8 of MAXPATH and on line 3 of MAXSPATH,
	for each vertex $v$, MAXSPATH($G, v$) is at most called once.
	Thus, there are in total at most $|V| = n$ MAXSPATH calls. 

	For each iteration in the for loop of MAXSPATH with outneighbour $u$, the
	run time is at most $3 + C(\mbox{MAXSPATH}(G,u))$.
	Since we have the constrain on the number of time a MAXSPATH$(G,v)$ is
	performed, and there might be multiple vertices sharing a same
	out-neighbour, here we just consider the run time of MAXSPATH($G,s$)
	subtract by all nested MAXSPATH($G, u$) calls. 
	Then, the run time of the for loop subtractiong all MAXSPATH($G, u$) calls
	is at most $|\{u \text{ out-neighbour of } s\}|$.
	Thus, in total, the run time of MAXSPATH with out all nested MAXSPATH is at
	most 
	$$2 + 3|\{u \text{ out-neighbour of } s\}|$$
	Note that for $s$ with no out-neighbours, the run time above is just the
	upper bound of its actual run time. 

	The run time for the first for loop in MAXPATH is exactly $|V|$. 

	Consider the second for loop from line 7 to 15.
	Similar what I did for the for loop in MAXSPATH, let's just consider the
	run time of this for loop subtracting the run time of all MAXSPATH. 
	Then, for each iteration, there are at most 4 operations exculding the
	MAXSPATH call. 
	Thus, in total, the run time is at most $4|V|$. 
	
	Now, consider the total run-time of all MAXSPATH($G, s$). 
	Recall that for each $v \in V$, MAXSPATH($G, v$) is at most performed once. 
	Then, we can sum the run time for each MAXSPATH($G, v$) without its nested
	MAXSPATH($G, v$) and add the run time of MAXSPATH($G,s$) for $s$ with no
	out-neighbours, and by this telescoping we can get the total run time.
	By the note above, this is just 
	$$\sum_{v \in V}(2+3|\{u \text{ out-neighbour of } v\}|) = 2|V| + 3|E|$$. 

	Therefore, the total run time of the for loop from line 7 to 15 is at most
	$6|V| + 3|E|$.

	Combine the first for loop and line 5, 6, 16, the total run time of
	MAXPATH($G$) is at most $7|V|+3|E|+3 \in O(|E|+|V|)$
\end{proof}


\end{document}

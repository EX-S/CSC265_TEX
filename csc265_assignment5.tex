\documentclass[10pt]{article}
\usepackage{amsmath}
\usepackage{amssymb}
\usepackage{fullpage}
\usepackage{comment}
\usepackage{graphicx}
\usepackage{listings}
\usepackage{enumitem}
\usepackage{scrextend}

\newcommand{\iimplies}{\mbox{ IMPLIES }}
\newcommand{\oor}{\mbox{ OR }}
\newcommand{\aand}{\mbox{ AND }}
\newcommand{\nnot}{\mbox{ NOT }}
\newcommand{\iiff}{\mbox{ IFF }}
\newcommand{\xxor}{\mbox{ XOR }}
%\newcommand{\algorithmicbreak}{\textbf{break}}
%\newcommand{\Break}{\State \algorithmicbreak}

\newtheorem{theorem}{Theorem}
\newtheorem{corollary}{Corollary}[theorem]
\newtheorem{lemma}[theorem]{Lemma}

\begin{document}

\begin{center}
{\bf \Large \bf CSC265H Assignment 5}
\end{center}

\noindent
Yibin Zhao\\
1002996261\\
NO OUTSIDE DISCUSSION\\
NO EXTRA MATERIAL CONSULTED\\

\begin{enumerate}
	\item 
		\begin{enumerate}
			\item
				Let $T$ be any Rtree.
				Let $T'$ be the resulting Rtree after applying some rebalancing
				rule on $T$. 
				
				\textit{Rule 1:}
				\begin{addmargin}[1em]{0em}
					There is a zero weight child of the root.
					The other might be zero or overwight.
					If $\Psi(T') > \Psi(T)$, then either $a(T)$ increment by 1
					or $c(T)$ increment by 1.
					Thus, the maximum increament is 4.
				\end{addmargin}
				\textit{Rule 2:}
				\begin{addmargin}[1em]{0em}
					The sibling of the 0 weight grandchild of $x$ might has 0
					weight or might be overweight.
					Also, $w-1$ could be 0.
					Since there are too many possible cases, I will only show
					the maximum case.
					Consider all grandchilren of $x$ have 0 weight and $x$ is overweight.
					Then $a(T') = a(T)+1, b(T') = b(T)+1, c(T') = c(T)$ and
					$\Psi(T') = \Psi(T) + 6$.
					Thus, the maximum increament is 6.
				\end{addmargin}
				\textit{Rule 3:}
				\begin{addmargin}[1em]{0em}
					Either (a) or (b), applying rule 3 will always increase the
					$a(T)$ by 1 and $b(T)$ always keeps the same.
					Since the sibling of $y$ might has a weight of 0, $c(T)$
					might not decrease.
					Therefore, the maximum increament in $\Psi$ is 4. 
				\end{addmargin}
				\textit{Rule 4:}
				\begin{addmargin}[1em]{0em}
					Since the weight of the root is still positive after
					applying the rule, and the stucture of the whole tree
					doesn't change, $\Psi(T') = \Psi(T)$.
				\end{addmargin}
				\textit{Rule 5:}
				\begin{addmargin}[1em]{0em}
					Since both children of $x$ are overweight initially, $\Psi$
					keeps the same after the transformation. 
				\end{addmargin}
				\textit{Rule 6:}
				\begin{addmargin}[1em]{0em}
					$a(T)$ remains the same, $b(T)$ decreases by 1.
					If $w = 0, w'' > 2$, then $c(T)$ increases by 1.
					However, $\Psi$ always decreases since $b(T)$ decreases.
				\end{addmargin}
				\textit{Rule 7:}
				\begin{addmargin}[1em]{0em}
					(a):
					\begin{addmargin}[1em]{0em}
						Consider after applying the rule.

						\textit{Case 1:} $u$'s two children have weight 0 and
						$y$'s sibling has weight 0.
						\begin{addmargin}[1em]{0em}
							In this case, $a(T') = a(T), b(T') = b(T)$.
							if $w' > 2$, then $c(T') = c(T)+1$.
							Thus, $\Psi$ at most increase by 1.
						\end{addmargin}
						\textit{Case 2:} $u$'s two children have weight 0 and
						$y$'s sibling has positive weight.
						\begin{addmargin}[1em]{0em}
							$a(T') = a(T)+1, b(T') = b(T)+1$
							If the sibling of $y$ has weight 1, then $c(T)$
							keeps the same value.
							Otherwise, $c(T)$ decreases by 1.
							Thus, $\Psi$ at most increases by 6.
						\end{addmargin}
						\textit{Otherwise:}
						\begin{addmargin}[1em]{0em}
							Since at least 1 of $u$'s child has positive
							weight, $a(T)$ can not increase.
							$b(T)$ can increase by at most 2, and $c(T)$ can
							increase by at most 2.
							Therefore, the increment on $\Psi$ is smaller or
							equal to 6.
						\end{addmargin}
						Combine all these cases, the maximum amount $\Psi$ can
						increase is 6.
					\end{addmargin}
					(b):
					\begin{addmargin}[1em]{0em}
						Consider after performing the transformation.

						\textit{Case 1:} $z$'s two children have weight 0, and
						$y$'s sibling has 0 weight.
						\begin{addmargin}[1em]{0em}
							Same as (a) Case 1.
						\end{addmargin}
						\textit{Case 2:} $z$'s two children have weight 0, and
						$y$'s sibling has positive weight.
						\begin{addmargin}[1em]{0em}
							Then, $a(T') = a(T)$, since the sibling of $y$ was
							a child of $u$ instead of $z$.
							$b(T') = b(T)+1$.
							$c(T') = c(T)$ if the sibling of $y$ has weight 1,
							or otherwise, $c(T') = c(T)-1$.
							Thus, $\Psi$ at most increases by 2.
						\end{addmargin}
						\textit{Case 3:} Otherwise.
						\begin{addmargin}[1em]{0em}
							$a(T')$ always less than or equal to $a(T)$. 
							Similar to (a) Casa 3, $b(T)$ at most increases by
							2 and $c(T)$ at most increases by 2.
							Thus, $\Psi$ at most increases by 6.
						\end{addmargin}
						Hence, $\Psi$ at most increases by 6.
					\end{addmargin}
					Combine both (a) and (b), we can conclude that the maximum
					increament on $\Psi$ is 6 for rule 7.
				\end{addmargin}
				\textit{Rule 8:}
				\begin{addmargin}[1em]{0em}
					In this case, $\Psi$ can not increase, since $c(T)$ always decrease by 1 and $a(T), b(T)$ keep their value.
				\end{addmargin}

				Combining all 8 reblancing rules, the maximum amount $\Psi$ can
				increase as the result of the application of one revalancing
				rule is 6. 
				Thus the upper bound $m=6$.

			\item
				By assignment 2, we already known that applying rule 1 to 7 at least decrease the number of violations in the Rtree by 1.
				Therefore, for rule 1 to 7, suppose $T'$ is the resulting Rtree, then $\Phi(T') = (m+1)v(T') + \Psi(T') \leq (m+1)(v(T) - 1) + (m + \Psi(T)) = (m+1)v(T) + \Psi(T) - 1$.
				$\Phi$ decreases by at least 1.

				Now consider applying the rule 8. 
				The number of violation in the Rtree might not decrease in this case. 
				But, still by assignment2, the number of violation would not increase.
				By (a), the $c(T)$ decrease by 1 and $a(T), b(T)$ keep their value.
				Thus, $\Psi(T)$ decrease by 1 after applying rule 8.
				Therefore, $\Phi(T') = (m+1)v(T') + \Psi(T') = (m+1)v(T') + \Psi(T)-1 \leq (m+1)v(T) + \Psi(T) - 1$ and $\Phi$ decreases by at least 1.

				Thus, combine rule 1 to 7 and rule 8, $\Phi$ decreases by at least 1.
			
			\item %TODO
				\begin{lemma}
					The total number of violations after performing any
					sequence $\sigma$ consisting of a total of $m$ insert
					and delete operations and the application of any number of
					rebalancing rules to an Rtree with one node is no greater
					than $m+w-1$, where $w$ is the weight of the node in the
					initial tree.
				\end{lemma}
				\begin{proof}
					Let $T_0$ be the initial Rtree.
					$T_i$ be the tree after the $i$th operation.
					Let $v_{i,j}$ be the number of violations in the Rtree after the
					$i$th operation, where $j$ is the number of insertion and
					deletion that have been performed. 
					Let $P(i,j)$ be the predicate "$v_{i,j} \leq j+w-1$", where $i
					\in \mathbb{N}, j = 0, 1, \cdots, m$.
					Note that $i \geq j$.
					
					\textbf{Base Case:} $j=0$
					\begin{addmargin}[1em]{0em}
						Since $T_0$ only has one node with a weight of $w$,
						there could only be overweight violations in $T_0$, and
						the total number of violations in this tree is $w-1$.
						Thus $P(0)$ holds.
					\end{addmargin}

					\textbf{Inductive Step:}
					\begin{addmargin}[1em]{0em}
						Suppose $P(i, j)$ is true for some $i$ and $j$, where
						$i \geq j$. 

					\end{addmargin}

				\end{proof}

				Since Rtree is lead oriented, an Rtree with exactly one leaf is an Rtree with only one node.
				Since deletion requires a pointer to an element in the Rtree, the total number of deletions in $\sigma$ is at most $\left\lceil \frac{m}{2} \right\rceil$.
				
				Let $T_i$ be the Rtree after the $i$th operation in sequence $\sigma$ on the Rtree.
				Let $c_i$ be the actual cost of the $i$th operation.
				Let $a_i$ be the amartized cost of the $i$the operation.

				Initially, $\Phi(T_0) = (m+1)v(T_0) + \Psi(T_0) = 0$, since there is only one node in the tree.
				Since, $v(T) \geq 0$ and $\Psi(T) \geq 0$ for any Rtree $T$, $\Phi(T) \geq 0$. 

				\textbf{INSERTION}
				\begin{addmargin}
					Suppose inserting under node $x$ with weight $w$. 
					Since the tree is leaf oriented tree, $x$ must be a leaf.
					Thus, $w \geq 1$ \\.
					\textit{Case 1:} $w = 1$
					\begin{addmargin}[1em]{0em}
						Then, $c(T_i) = c(T_{i-1})$.

						If $a(T_{i-1})$ increses by 1, in which case $b(T_i) = b(T_{i-1})$, and the number of violation maintains its value.

						Otherwise, $a(T_i) = a(T_{i-1})$.
						If $x$ has no parent, then $b(T_i) = b (T_{i-1})$.

						If $x$ has a parent, then $x$ must have a sibling, say $y$.
						$y$ has either 0 weight or 1 weight by the properties of an Rtree.
						Since we assume $a(T_i) = a(T_{i-1})$ in this case, then either the parent of $x$ has 0 weight or $y$ has a weight of 1.
						%TODO
					\end{addmargin}
					\textit{Case 2:} $w \geq 1$
					\begin{addmargin}[1em]{0em}
						%TODO
					\end{addmargin}
				\end{addmargin}
				
				\textbf{DELETION}
				%TODO

				\textbf{REBALANCING}
				\begin{addmargin}
				Let the true cost of a single rotation be $r$.
				Then a double rotation has a cost of $2r$.
				Let the change of weight on a node with a true cost of 1.
				Then, rule 1 has a true cost of 1, rule 2 has a true cost of 3, rule 3(a) has a true cost of $r$, rule 3(b) has a true cost of $2r$, 
				rule 4 has a true cost of 1, rule 5 has a true cost of 3, rule 6 has a true cost of 3, rule 7(a) has a true cost of $r+3$, 7(b) has a ture cost of $2r=3$,
				and rule 8 has a true cost of $r+2$.
				Therefore, the upperbound of the true cost $c_i$ of applying a rebalancing rule  is $2r+3$.
				By (b), we know that $\Phi(T_i) - \Phi(T_{i-1}) \leq -1$.
				Then $a_i = c_i + \Phi(T_i) - \Phi(T_{i-1}) \leq 2r+3 + (-1) = 2r+2$
				\end{addmargin}
		\end{enumerate}
\end{enumerate}

\end{document}

\documentclass[10pt]{article}
\usepackage{amsmath}
\usepackage{amssymb}
\usepackage{fullpage}
\usepackage{comment}
\usepackage{graphicx}
\usepackage{listings}
\usepackage{enumitem}
\usepackage{scrextend}

\newcommand{\iimplies}{\mbox{ IMPLIES }}
\newcommand{\oor}{\mbox{ OR }}
\newcommand{\aand}{\mbox{ AND }}
\newcommand{\nnot}{\mbox{ NOT }}
\newcommand{\iiff}{\mbox{ IFF }}
\newcommand{\xxor}{\mbox{ XOR }}
%\newcommand{\algorithmicbreak}{\textbf{break}}
%\newcommand{\Break}{\State \algorithmicbreak}

\begin{document}

\begin{center}
{\bf \Large \bf CSC265H Assignment 5}
\end{center}

\noindent
Yibin Zhao\\
1002996261\\
NO OUTSIDE DISCUSSION\\
NO EXTRA MATERIAL CONSULTED\\

\begin{enumerate}
	\item 
		\begin{enumerate}
			\item
				Let $T$ be any Rtree.
				Let $T'$ be the resulting Rtree after applying some rebalancing
				rule on $T$. 
				
				\textit{Rule 1:}
				\begin{addmargin}[1em]{0em}
					There is a zero weight child of the root.
					The other might be zero or overwight.
					If $\Psi(T') > \Psi(T)$, then either $a(T)$ increment by 1
					or $c(T)$ increment by 1.
					Thus, the maximum increament is 4.
				\end{addmargin}
				\textit{Rule 2:}
				\begin{addmargin}[1em]{0em}
					The sibling of the 0 weight grandchild of $x$ might has 0
					weight or might be overweight.
					Also, $w-1$ could be 0.
					Since there are too many possible cases, I will only show
					the maximum case.
					Consider all grandchilren of $x$ have 0 weight and $x$ is overweight.
					Then $a(T') = a(T)+1, b(T') = b(T)+1, c(T') = c(T)$ and
					$\Psi(T') = \Psi(T) + 6$.
					Thus, the maximum increament is 6.
				\end{addmargin}
				\textit{Rule 3:}
				\begin{addmargin}[1em]{0em}
					Either (a) or (b), applying rule 3 will always increase the
					$a(T)$ by 1 and $b(T)$ always keeps the same.
					Since the sibling of $y$ might has a weight of 0, $c(T)$
					might not decrease.
					Therefore, the maximum increament in $\Psi$ is 4. 
				\end{addmargin}
				\textit{Rule 4:}
				\begin{addmargin}[1em]{0em}
					Since the weight of the root is still positive after
					applying the rule, and the stucture of the whole tree
					doesn't change, $\Psi(T') = \Psi(T)$.
				\end{addmargin}
				\textit{Rule 5:}
				\begin{addmargin}[1em]{0em}
					Since both children of $x$ are overweight initially, $\Psi$
					keeps the same after the transformation. 
				\end{addmargin}
				\textit{Rule 6:}
				\begin{addmargin}[1em]{0em}
					$a(T)$ remains the same, $b(T)$ decreases by 1.
					If $w = 0, w'' > 2$, then $c(T)$ increases by 1.
					However, $\Psi$ always decreases since $b(T)$ decreases.
				\end{addmargin}
				\textit{Rule 7:}
				\begin{addmargin}[1em]{0em}
					(a):
					\begin{addmargin}[1em]{0em}
						Consider after applying the rule.

						\textit{Case 1:} $u$'s two children have weight 0 and
						$y$'s sibling has weight 0.
						\begin{addmargin}[1em]{0em}
							In this case, $a(T') = a(T), b(T') = b(T)$.
							if $w' > 2$, then $c(T') = c(T)+1$.
							Thus, $\Psi$ at most increase by 1.
						\end{addmargin}
						\textit{Case 2:} $u$'s two children have weight 0 and
						$y$'s sibling has positive weight.
						\begin{addmargin}[1em]{0em}
							$a(T') = a(T)+1, b(T') = b(T)+1$
							If the sibling of $y$ has weight 1, then $c(T)$
							keeps the same value.
							Otherwise, $c(T)$ decreases by 1.
							Thus, $\Psi$ at most increases by 6.
						\end{addmargin}
						\textit{Otherwise:}
						\begin{addmargin}[1em]{0em}
							Since at least 1 of $u$'s child has positive
							weight, $a(T)$ can not increase.
							$b(T)$ can increase by at most 2, and $c(T)$ can
							increase by at most 2.
							Therefore, the increment on $\Psi$ is smaller or
							equal to 6.
						\end{addmargin}
						Combine all these cases, the maximum amount $\Psi$ can
						increase is 6.
					\end{addmargin}
					(b):
					\begin{addmargin}[1em]{0em}
						Consider after performing the transformation.

						\textit{Case 1:} $z$'s two children have weight 0, and
						$y$'s sibling has 0 weight.
						\begin{addmargin}[1em]{0em}
							Same as (a) Case 1.
						\end{addmargin}
						\textit{Case 2:} $z$'s two children have weight 0, and
						$y$'s sibling has positive weight.
						\begin{addmargin}[1em]{0em}
							Then, $a(T') = a(T)$, since the sibling of $y$ was
							a child of $u$ instead of $z$.
							$b(T') = b(T)+1$.
							$c(T') = c(T)$ if the sibling of $y$ has weight 1,
							or otherwise, $c(T') = c(T)-1$.
							Thus, $\Psi$ at most increases by 2.
						\end{addmargin}
						\textit{Case 3:} Otherwise.
						\begin{addmargin}[1em]{0em}
							$a(T')$ always less than or equal to $a(T)$. 
							Similar to (a) Casa 3, $b(T)$ at most increases by
							2 and $c(T)$ at most increases by 2.
							Thus, $\Psi$ at most increases by 6.
						\end{addmargin}
						Hence, $\Psi$ at most increases by 6.
					\end{addmargin}
					Combine both (a) and (b), we can conclude that the maximum
					increament on $\Psi$ is 6 for rule 7.
				\end{addmargin}
				\textit{Rule 8:}
				\begin{addmargin}[1em]{0em}
					In this case, $\Psi$ can not increase.
				\end{addmargin}

				Combining all 8 reblancing rules, the maximum amount $\Psi$ can
				increase as the result of the application of one revalancing
				rule is 6. 
				Thus the upper bound $m=6$.

			\item %TODO
			\item %TODO
		\end{enumerate}
\end{enumerate}

\end{document}

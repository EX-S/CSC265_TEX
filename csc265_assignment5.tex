\documentclass[10pt]{article}
\usepackage{amsmath}
\usepackage{amssymb}
\usepackage{amsthm}
\usepackage{fullpage}
\usepackage{comment}
\usepackage{graphicx}
\usepackage{listings}
\usepackage{enumitem}
\usepackage{scrextend}

\newcommand{\iimplies}{\mbox{ IMPLIES }}
\newcommand{\oor}{\mbox{ OR }}
\newcommand{\aand}{\mbox{ AND }}
\newcommand{\nnot}{\mbox{ NOT }}
\newcommand{\iiff}{\mbox{ IFF }}
\newcommand{\xxor}{\mbox{ XOR }}
%\newcommand{\algorithmicbreak}{\textbf{break}}
%\newcommand{\Break}{\State \algorithmicbreak}

\newtheorem{theorem}{Theorem}
\newtheorem{corollary}{Corollary}[theorem]
\newtheorem{lemma}[theorem]{Lemma}

\begin{document}

\begin{center}
{\bf \Large \bf CSC265H Assignment 5}
\end{center}

\noindent
Yibin Zhao\\
1002996261\\
NO OUTSIDE DISCUSSION\\
NO EXTRA MATERIAL CONSULTED\\

\begin{enumerate}
	\item 
		\begin{enumerate}
			\item
				Let $T$ be any Rtree.
				Let $T'$ be the resulting Rtree after applying some rebalancing
				rule on $T$. 
				
				\textit{Rule 1:}
				\begin{addmargin}[1em]{0em}
					There is a zero weight child of the root.
					The other might be zero or overwight.
					If $\Psi(T') > \Psi(T)$, then either $a(T)$ increment by 1
					or $c(T)$ increment by 1.
					Thus, the maximum increament is 4.
				\end{addmargin}
				\textit{Rule 2:}
				\begin{addmargin}[1em]{0em}
					The sibling of the 0 weight grandchild of $x$ might has 0
					weight or might be overweight.
					Also, $w-1$ could be 0.
					Since there are too many possible cases, I will only show
					the maximum case.
					Consider all grandchilren of $x$ have 0 weight and $x$ is overweight.
					Then $a(T') = a(T)+1, b(T') = b(T)+1, c(T') = c(T)$ and
					$\Psi(T') = \Psi(T) + 6$.
					Thus, the maximum increament is 6.
				\end{addmargin}
				\textit{Rule 3:}
				\begin{addmargin}[1em]{0em}
					Either (a) or (b), applying rule 3 will always increase the
					$a(T)$ by 1 and $b(T)$ always keeps the same.
					Since the sibling of $y$ might has a weight of 0, $c(T)$
					might not decrease.
					Therefore, the maximum increament in $\Psi$ is 4. 
				\end{addmargin}
				\textit{Rule 4:}
				\begin{addmargin}[1em]{0em}
					Since the weight of the root is still positive after
					applying the rule, and the stucture of the whole tree
					doesn't change, $\Psi(T') = \Psi(T)$.
				\end{addmargin}
				\textit{Rule 5:}
				\begin{addmargin}[1em]{0em}
					Since both children of $x$ are overweight initially, $\Psi$
					keeps the same after the transformation. 
				\end{addmargin}
				\textit{Rule 6:}
				\begin{addmargin}[1em]{0em}
					$a(T)$ remains the same, $b(T)$ decreases by 1.
					If $w = 0, w'' > 2$, then $c(T)$ increases by 1.
					However, $\Psi$ always decreases since $b(T)$ decreases.
				\end{addmargin}
				\textit{Rule 7:}
				\begin{addmargin}[1em]{0em}
					(a):
					\begin{addmargin}[1em]{0em}
						Consider after applying the rule.

						\textit{Case 1:} $u$'s two children have weight 0 and
						$y$'s sibling has weight 0.
						\begin{addmargin}[1em]{0em}
							In this case, $a(T') = a(T), b(T') = b(T)$.
							if $w' > 2$, then $c(T') = c(T)+1$.
							Thus, $\Psi$ at most increase by 1.
						\end{addmargin}
						\textit{Case 2:} $u$'s two children have weight 0 and
						$y$'s sibling has positive weight.
						\begin{addmargin}[1em]{0em}
							$a(T') = a(T)+1, b(T') = b(T)+1$
							If the sibling of $y$ has weight 1, then $c(T)$
							keeps the same value.
							Otherwise, $c(T)$ decreases by 1.
							Thus, $\Psi$ at most increases by 6.
						\end{addmargin}
						\textit{Otherwise:}
						\begin{addmargin}[1em]{0em}
							Since at least 1 of $u$'s child has positive
							weight, $a(T)$ can not increase.
							$b(T)$ can increase by at most 2, and $c(T)$ can
							increase by at most 2.
							Therefore, the increment on $\Psi$ is smaller or
							equal to 6.
						\end{addmargin}
						Combine all these cases, the maximum amount $\Psi$ can
						increase is 6.
					\end{addmargin}
					(b):
					\begin{addmargin}[1em]{0em}
						Consider after performing the transformation.

						\textit{Case 1:} $z$'s two children have weight 0, and
						$y$'s sibling has 0 weight.
						\begin{addmargin}[1em]{0em}
							Same as (a) Case 1.
						\end{addmargin}
						\textit{Case 2:} $z$'s two children have weight 0, and
						$y$'s sibling has positive weight.
						\begin{addmargin}[1em]{0em}
							Then, $a(T') = a(T)$, since the sibling of $y$ was
							a child of $u$ instead of $z$.
							$b(T') = b(T)+1$.
							$c(T') = c(T)$ if the sibling of $y$ has weight 1,
							or otherwise, $c(T') = c(T)-1$.
							Thus, $\Psi$ at most increases by 2.
						\end{addmargin}
						\textit{Case 3:} Otherwise.
						\begin{addmargin}[1em]{0em}
							$a(T')$ always less than or equal to $a(T)$. 
							Similar to (a) Casa 3, $b(T)$ at most increases by
							2 and $c(T)$ at most increases by 2.
							Thus, $\Psi$ at most increases by 6.
						\end{addmargin}
						Hence, $\Psi$ at most increases by 6.
					\end{addmargin}
					Combine both (a) and (b), we can conclude that the maximum
					increament on $\Psi$ is 6 for rule 7.
				\end{addmargin}
				\textit{Rule 8:}
				\begin{addmargin}[1em]{0em}
					In this case, $\Psi$ can not increase, since $c(T)$ always decrease by 1 and $a(T), b(T)$ keep their value.
				\end{addmargin}

				Combining all 8 reblancing rules, the maximum amount $\Psi$ can
				increase as the result of the application of one revalancing
				rule is 6. 
				Thus the upper bound $m=6$.

			\item
				By assignment 2, we already known that applying rule 1 to 7 at least decrease the number of violations in the Rtree by 1.
				Therefore, for rule 1 to 7, suppose $T'$ is the resulting Rtree, then $\Phi(T') = (m+1)v(T') + \Psi(T') \leq (m+1)(v(T) - 1) + (m + \Psi(T)) = (m+1)v(T) + \Psi(T) - 1$.
				$\Phi$ decreases by at least 1.

				Now consider applying the rule 8. 
				The number of violation in the Rtree might not decrease in this case. 
				But, still by assignment2, the number of violation would not increase.
				By (a), the $c(T)$ decrease by 1 and $a(T), b(T)$ keep their value.
				Thus, $\Psi(T)$ decrease by 1 after applying rule 8.
				Therefore, $\Phi(T') = (m+1)v(T') + \Psi(T') = (m+1)v(T') + \Psi(T)-1 \leq (m+1)v(T) + \Psi(T) - 1$ and $\Phi$ decreases by at least 1.

				Thus, combine rule 1 to 7 and rule 8, $\Phi$ decreases by at least 1.
			
			\item 
				Let $T_i$ be the Rtree after the $i$th operation in sequence $\sigma$ on the Rtree.
				Let $c_i$ be the actual cost of the $i$th operation.
				Let $a_i$ be the amartized cost of the $i$the operation.

				Initially, $\Phi(T_0) = (m+1)v(T_0) + \Psi(T_0) = 0$, since
				there is only one node with weight 1.
				Since, $v(T) \geq 0$ and $\Psi(T) \geq 0$ for any Rtree $T$, $\Phi(T) \geq 0$.

				\begin{lemma}
					The total number of operations in sequence $\sigma$ is in $O(k)$.
				\end{lemma}
				\begin{proof}
					Since each application of one rebalancing rule decreases
					$\Phi$ by 1.
					Suppose $\Phi(T) = 0$ for some Rtree $T$, then no
					rebalacing rule could be applied.

					Consider the increment of $\Phi$ performing an insertion on an Rtree.
					$a$ might increase by 1 and $b$ might increase by 1, $c$
					can not increase. 
					Thus, $\Psi$ could change by at most 6. 
					The number of 0-0 violation could increase by 1, and
					therefore, the increment of $\Phi$ is at most $(6+1)\times1
					+ 6 = 13$.

					Next, consider the increment for deletion.
					$a$ can not increase. $b$ could increase by 1 and $c$ as
					well. 
					It follows that $\Psi$ can only increase by at most 3.
					No 0-0 violations would be caused by deletion. 
					If $w''$ is 0, then the number of overweight violations
					won't change.
					Otherwise, overweight violation increases by $(w+w''-1) -
					(w-1 + w''-1) = 1$.
					Hence, $\Phi$ increases by at most $(6+1) \times 1 + 3 =
					10$.

					Therefore, the increament of $\Phi$ is smaller or equal to
					13 for an insertion or deletion operation.

					Let $T_{u,v} = T_i$, where $u = 0, 1, \cdots, k$ is the number of insertion
					and deletion applied, $v \in \mathbb{N}$ is the total number of
					applications of rebalancing rules and $i = u + v$. 

					Let $P(u,v)$ be the predicate "$\Phi(T_{u,v}) \leq 13u-v$".
					We want to prove that $\forall u \in \{0, 1, \cdots, k\}.
					\forall k \in \mathbb{N}. P(u,v)$. Perform double induction
					to prove this.

					\textbf{Base Case:} $u = 0$
					\begin{addmargin}[1em]{0em}
						Base Case: $v = 0$
						\begin{addmargin}[1em]{0em}
							Since $\Phi(T_{0, 0}) = \Phi(T_0) = 0$, $P(0,0)$
							holds.
						\end{addmargin}
						
						Inductive Step:
						\begin{addmargin}[1em]{0em}
							Suppose $P(0, v)$ is true for some $v \in
							\mathbb{N}$.
							Then $\Phi(T_{0, v}) \leq 13u-v$.
							By (b), $\Phi(T_{0,v+1}) \leq \Phi(T_{0, v}) - 1
							\leq 13u-(v+1)$.
							Thus $P(u, v+1)$ is true.
						\end{addmargin}

						By induction, $\forall k \in \mathbb{N}. P(0, v)$ is
						true.
					\end{addmargin}

					\textbf{Inductive Step:}
					\begin{addmargin}[1em]{0em}
						Suppose $\forall vk \in \mathbb{N}. P(u,v)$ is true for
						some $u \in \{0, 1, \cdots, k-1\}$.
						Then, for any $v \in \mathbb{N} \Phi(T_{u+1, v+1}) \leq
						\max\{\Phi(T_{u,v+1})+13, \Phi(T_{u+1,v})-1\}$.
						Recall that 13 is the upper bound of the increment of
						$\Phi$ for performing a single insertion or deletion.

						Base Case: $vk = 0$
						\begin{addmargin}[1em]{0em}
							Then, $\Phi(T_{u+1,0}) \leq \Phi(T_{u, 0})+13 \leq
							13(u+1)$.
							Thus, $P(u+1,0)$ holds.
						\end{addmargin}

						Inductive Step:
						\begin{addmargin}[1em]{0em}
							Suppose $P(u+1, v)$ is true for some $v \in
							\mathbb{N}$.
							Then $P(u+1, v) \leq 13(u+1)-v$.
							$\Phi(T_{u,v+1})+13 \leq 13u-(v+1)+13 =
							13(u+1)-(v+1)$.
							$\Phi(T_{u+1,v})-1 \leq 13(u+1)-v-1 =
							13(u+1)-(v+1)$.
							Therefore, $\Phi(T_{u+1, v+1}) \leq
							\max\{\Phi(T_{u,v+1})+13, \Phi(T_{u+1,v})-1\} \leq
							13(u+1)-(v+1)$.
							Hence, $P(u+1,v+1)$ is true.
						\end{addmargin}

						By induction, $\forall k \in \mathbb{N}. P(u+1, v)$ is
						true.

					\end{addmargin}

					By double induction, $\forall u \in \{0, 1, \cdots, k\}.
					\forall k \in \mathbb{N}. P(u, v)$.

					Then $P(k, v)$ is true for any $v \in \mathbb{N}$.
					That is, $\Phi(T_{k,v}) \leq 13k-v$, for all natural number
					$v$.
					Since $\Phi \geq 0$, $13k-v \geq 0$, which implies that $v
					\leq 13k$. 
					Thus the total number of operations is bounded by $14k$, which
					is in $O(k)$. 
				\end{proof}

				Now consider the amartized cost.
				Consider any $i$th operation in the sequence $\sigma$.

				\textbf{INSERTION}
				\begin{addmargin}[1em]{0em}
					$\Phi(T_i) - \Phi(T_{i-1}) \leq 13$ as shown in the proof
					of the lemma.
					$c_i = 4$ for initializing and connecting two new node. 
					Then $a_i = c_i + \Phi(T_i) - \Phi(T_{i-1}) \leq 17$.
				\end{addmargin}

				\textbf{DELETION}
				\begin{addmargin}[1em]{0em}
					$\Phi(T_i) - \Phi(T_{i-1}) \leq 10$, also shown in the
					proof of lemma.
					$c_i = 4$ for removing 2 nodes, changing weight, and
					modifying a pointer.
					Then $a_i = c_i + \Phi(T_i) - \Phi(T_{i-1}) \leq 14$
				\end{addmargin}

				\textbf{REBALANCING}
				\begin{addmargin}[1em]{0em}
					Let the true cost of a single rotation be some constant $r$.
					Then a double rotation has a cost of $2r$.

					Rule 1 has a true cost of 1, rule 2 has a true cost
					of 3, rule 3(a) has a true cost of $r$, rule 3(b) has a
					true cost of $2r$, 	rule 4 has a true cost of 1, rule 5 has
					a true cost of 3, rule 6 has a true cost of 3, rule 7(a)
					has a true cost of $r+3$, 7(b) has a true cost of $2r+3$, 
					and rule 8 has a true cost of $r+2$.

					Therefore, the upperbound of the true cost $c_i$ of
					applying a rebalancing rule is $2r+3$. 
					By (b), we know that $\Phi(T_i) - \Phi(T_{i-1}) \leq -1$.
					Then $a_i = c_i + \Phi(T_i) - \Phi(T_{i-1}) \leq 2r+3 + (-1) = 2r+2$
				\end{addmargin}

				Hence, the amartized cost of one operation is bounded by
				$\max\{17, 14, 2r+2\}$, which is in $O(1)$.
				Since the total number of operations is in $O(k)$ and the
				amartized cost of each operation is in $O(1)$, the total cost
				of $\sigma$ is $O(k)$. 
		\end{enumerate}
\end{enumerate}

\end{document}

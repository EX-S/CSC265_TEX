\documentclass[10pt]{article}
\usepackage{amsmath}
\usepackage{amssymb}
\usepackage{amsthm}
\usepackage{fullpage}
\usepackage{comment}
\usepackage{graphicx}
\usepackage{listings}
\usepackage{enumitem}
\usepackage{scrextend}

\newcommand{\iimplies}{\mbox{ IMPLIES }}
\newcommand{\oor}{\mbox{ OR }}
\newcommand{\aand}{\mbox{ AND }}
\newcommand{\nnot}{\mbox{ NOT }}
\newcommand{\iiff}{\mbox{ IFF }}
\newcommand{\xxor}{\mbox{ XOR }}
%\newcommand{\algorithmicbreak}{\textbf{break}}
%\newcommand{\Break}{\State \algorithmicbreak}

\newtheorem{theorem}{Theorem}
\newtheorem{corollary}{Corollary}[theorem]
\newtheorem{lemma}[theorem]{Lemma}

\begin{document}

\begin{center}
{\bf \Large \bf CSC265H Assignment 8}
\end{center}

\noindent
Yibin Zhao\\
1002996261\\
NO OUTSIDE DISCUSSION\\
NO EXTRA MATERIAL CONSULTED\\

\begin{comment}
A black&white heap is a singly-linked list of binomial trees that satisfies the
following properties:
	The roots of the binomial trees in the list have strictly increasing
	degrees.
	The root of every binomial tree is white.
	If a white node is not the root of a binomial tree, its priority is greater
	than or equal to the priority of its parent.
	The black nodes all have different degrees.
	The degree od a black node is 1 less than the degree of its parent, i.e.
	the black node is the first child of its parent.

Recal that a node in a binomial heap has degree k if and only if the subtree
rooted at that node is a binomial tree with $2^k$ nodes.

We say that a linked-list of binomial trees is a k-black&while heap if it
satisfies all the properties of a black&white heap except that it has at
most two black nodes of degree k and at most one of its black nodes of degree k
has a parent of degree greater than $k+1$.
Thus for every natural number k, every black&white heap is a k-black&white
heap.
\end{comment}

\begin{enumerate}
	\item
	\begin{comment}
		Suppose you have a linked-list H of binomial trees that satisfies all
		the properties of a black&white heap except that it has two black nodes
		$b$ and $b'$ of degree k both of which have parents of degree $k+1$.
		Given pointers to $b$ and $b'$, explain how to transform H in constant
		time into a k'-black&white heap with the same set of nodes, for some
		$k' > k$.
	\end{comment}

	\item
	\begin{comment}
		Suppose you have a linked-list H of binomial trees that satisfies all
		the properties of a black&white heap except that it has one black node
		$b$ of degree $k$ and its parent has degree greater than $k+1$.
		Suppose that the sibling $s$ of degree $k+1$ of $b$ is white.
		Given pointers to $b$ and $s$, explain how to transform H in constant
		time into a k'-black&white heap with the same set of nodes, for some
		$k' > k$.
	\end{comment}

	\item
	\begin{comment}
		Suppose you have a linked-list H of binomial trees that satisfies all
		the properties of a black&white heap except that it has one black node
		b od degree k and its parent has degree greater than $k+1$.
		Suppose that the sibling $s$ of degree $k+1$ of $b$ is black.
		Given pointers to $b$ and $s$, explain how to transform H in constant
		time into a k'-black&white heap with the some set of nodes, for some
		$k' > k$.
	\end{comment}

	\item
	\begin{comment}
		Suppose you have a linked-list H of binomial trees that satisfies all
		the properties of a black&white heap except that it had one black node
		$b$ of degree $k$ whose parent has degree greater than $k+1$ and
		another black node of degree $k$ whose parent has degree $k+1$.
		Given pointers to $b$ and its sibling $s$ of degree $k+1$, explain how
		to transform H in constant time into a k'-black&white heap with the
		same set of nodes, for some $k' > k$.
	\end{comment}

	\item
	\begin{comment}
		The operation DECREASE-KEY(H, x, v) takes a node $x$ in a black&white
		heap H and a value $v$ and decrease the priority of $x$ by $v$. 
		Explain how to perform DECREASE-KEY(H, x, v) in a black&white heap so
		that its amortized cost in $O(1)$.
		Use the number of black nodes in the black&white heap as your potential
		function or, equivalently, maintain the invariant that there is one
		token on each black node.
	\end{comment}

\end{enumerate}

\end{document}

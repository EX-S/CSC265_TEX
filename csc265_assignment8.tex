\documentclass[10pt]{article}
\usepackage{amsmath}
\usepackage{amssymb}
\usepackage{amsthm}
\usepackage{fullpage}
\usepackage{comment}
\usepackage{graphicx}
\usepackage{listings}
\usepackage{enumitem}
\usepackage{scrextend}
\usepackage{algorithm}
\usepackage{algpseudocode}

\newcommand{\iimplies}{\mbox{ IMPLIES }}
\newcommand{\oor}{\mbox{ OR }}
\newcommand{\aand}{\mbox{ AND }}
\newcommand{\nnot}{\mbox{ NOT }}
\newcommand{\iiff}{\mbox{ IFF }}
\newcommand{\xxor}{\mbox{ XOR }}
%\newcommand{\algorithmicbreak}{\textbf{break}}
%\newcommand{\Break}{\State \algorithmicbreak}

\newtheorem{theorem}{Theorem}
\newtheorem{corollary}{Corollary}[theorem]
\newtheorem{lemma}[theorem]{Lemma}

\begin{document}

\begin{center}
{\bf \Large \bf CSC265H Assignment 8}
\end{center}

\noindent
Yibin Zhao\\
1002996261\\
NO OUTSIDE DISCUSSION\\
NO EXTRA MATERIAL CONSULTED\\

\begin{comment}
A black$&$white heap is a singly-linked list of binomial trees that satisfies the
following properties:
	The roots of the binomial trees in the list have strictly increasing
	degrees.
	The root of every binomial tree is white.
	If a white node is not the root of a binomial tree, its priority is greater
	than or equal to the priority of its parent.
	The black nodes all have different degrees.
	The degree od a black node is 1 less than the degree of its parent, i.e.
	the black node is the first child of its parent.

Recal that a node in a binomial heap has degree k if and only if the subtree
rooted at that node is a binomial tree with $2^k$ nodes.

We say that a linked-list of binomial trees is a k-black$&$while heap if it
satisfies all the properties of a black$&$white heap except that it has at
most two black nodes of degree k and at most one of its black nodes of degree k
has a parent of degree greater than $k+1$.
Thus for every natural number k, every black$&$white heap is a k-black$&$white
heap.
\end{comment}

\begin{enumerate}
	\item
	\begin{comment}
		Suppose you have a linked-list H of binomial trees that satisfies all
		the properties of a black$&$white heap except that it has two black nodes
		$b$ and $b'$ of degree k both of which have parents of degree $k+1$.
		Given pointers to $b$ and $b'$, explain how to transform H in constant
		time into a k'-black$&$white heap with the same set of nodes, for some
		$k' > k$.
	\end{comment}
		
		\begin{algorithmic}[1]
			\If{priority(parent($b$)) $\leq$ priority($b$)}
				\State colour $b$ white
			\ElsIf{priority(parent($b'$)) $\leq$ priority($b'$)}
				\State colour $b'$ white
			\Else
				\State swap $b$ and parent($b$)
				\If(parent($b$) is a root)
					\State colour parent($b$) white
				\EndIf
			\EndIf
		\end{algorithmic}

		By "swap" I mean swapping the value,priority as well as colour of the
		two nodes. 
		This could be done in constant time.

		Notice that $b$ and $b'$ cannot share the same parent, or the
		subtree rooted at their parent would not be a binomial tree.

		If the priority of the parent of $b$ is less than the priority of $b$
		itself, then we can safely colour $b$ white and still satisfy the
		third property. 
		Same thing could be done for $b'$.
		Then, after this operation, the number of black node of degree $k$ is
		1, and thus the binomial heap is transformed into a $k+1$-black$\&$white
		heap.

		Consider both $b$ and $b'$ has priorities less than their parents'
		respectively.
		Here I choose to swap $b$ with its parent.
		After this operation, the node $b$ now stores the values of parent($b$)
		and parent($b$) stores informations of $b$.
		If parent($b$) is a root, in order to maintain the second property, $b$
		should be coloured white. 
		Note that, now, all children of parent($b$) is white and their priority
		are	greater than or equal to the priority of the initial parent of $b$, 
		which is also greater than the now priority(parent($b$)). 
		Thus, we can safely colour it white.
		By the condition, there is at most 1 black node with degree $k+1$
		initially.
		After the operation, there are at most 2 black nodes with degree $k+1$
		and 1 black node with degree $k$.
		Therefore the binomial heap is transformed into a $k+1$-black$\&$white
		heap.

		Hence, we can always transform such a $k$-black$\&$white heap $H$ into a
		$k+1$-black$\&$white heap.

		Each line in this algorithm could be done in constant time and there is
		no loop in this algotithm.
		Thus, the whole algorithm could be done in constant time.

	\item
	\begin{comment}
		Suppose you have a linked-list H of binomial trees that satisfies all
		the properties of a black$&$white heap except that it has one black node
		$b$ of degree $k$ and its parent has degree greater than $k+1$.
		Suppose that the sibling $s$ of degree $k+1$ of $b$ is white.
		Given pointers to $b$ and $s$, explain how to transform H in constant
		time into a k'-black$&$white heap with the same set of nodes, for some
		$k' > k$.
	\end{comment}

		\begin{algorithmic}[1]
			\If{priority(parent($b$)) $\leq$ priority($b$)}
				\State colour $b$ white
			\Else
				\State swap $b$.nextsibling and $s$.firstchild.nextsibling
				\State swap $b$.parent and $s$.firstchild.parent
				\State swap $s$.nextsibling and $s$.firstchild
			\EndIf
		\end{algorithmic}

		The swaps here are different from the swap in 1.
		They only swap between two pointers.

		If the priority of the parent of $b$ is less than or equal to the
		priority of $b$, then $b$ could be coloured white without violating the
		third property.
		Then, the number of black nodes of degree $k$ has a parent of degree
		greater than $k+1$ is 0.
		Since all other black nodes are unchanged, the new heap is a
		$k+1$-black$\&$white heap.

		Otherwise, we can infer the inequlity: priority($b$) <
		priority(parent($b$)) $\leq$ priority($s$) $\leq$
		priority($s$.firstchild).
		Thus, the priorities of all white children of $s$ are greater than the
		priority of $b$.
		$k+1$ is at least 1, which means that $s$ always has some child.
		The first child of $s$ is of degree $k$, which means that it could not
		be black, or it violates the given condition. 
		Then, we can swap the subtree rooted at $b$ and the subtree rooted at
		the first child of $s$ and still keep the whole tree a binomial tree. 
		This is done by the three swap operations on pointers. 
		Notice that initially the next sibling of $s$ is just $b$.
		Then the three swap operations are correct.
		Note that the colour of $b$ could not be changed due to the inequality
		above.
		Also, the colour of the initial $s$.firstchild need not be changed.
		Now, there is still only one black node of order $k$, which is $b$,
		while, the parent of $b$ now has a degree of $k+1$.
		Therefore, the new heap is a $k+1$-black$\&$white heap.

		Hence, by the two cases above, we can always transform such a
		$k$-black$\&$white heap $H$ into a $k+1$-black$\&$white heap.

		Each operation could be done in constant time and there is no loop in
		this algorithm.
		Hence, it runs in constant time.

	\item
	\begin{comment}
		Suppose you have a linked-list H of binomial trees that satisfies all
		the properties of a black$&$white heap except that it has one black node
		b od degree k and its parent has degree greater than $k+1$.
		Suppose that the sibling $s$ of degree $k+1$ of $b$ is black.
		Given pointers to $b$ and $s$, explain how to transform H in constant
		time into a k'-black$&$white heap with the some set of nodes, for some
		$k' > k$.
	\end{comment}

		\begin{algorithmic}[1]
			\If{priority(parent($b$)) $\leq$ priority($b$)}
				\State colour $b$ white
			\ElsIf{priority(parent($b$)) $\leq$ priority($s$)}
				\State colour $s$ white
				\State perform algorithm in 2.
			\Else
				\State swap $b$ and parent($b$)
				\If(parent($b$) is a root)
					\State colour parent($b$) white
				\EndIf
			\EndIf
		\end{algorithmic}

		The swap operation here is the same as the one in 1.

		If the priority of the parent of $b$ is less than or equal to the
		priority of $b$, then we can make $b$ white and it satisfies the third
		property.
		After this operation, there is no node of degree $k$ with a parent of
		degree greater than $k+1$.
		Thus, given the condition in this question, the new heap is a
		$k+1$-black$\&$white heap. 

		Suppose the priority of the parent of $b$ is less than or equal to the
		priority of $s$.
		Notice the the parent of $b$ is also the parent of $s$.
		Then, Similar as the above case, we can make $s$ a white node. 
		Now, the heap satisfies the condition in question 2, and thus, we can
		perform the algorithm in 2, which gives us a $k+1$-black$\&$white heap.

		Otherwise, both $s$ and $b$ have priorities less than their parent.
		Since each white child of the parent has a priority greater than it,
		the priority of $b$ is less than any of its white siblings.
		Therefore, we can swap $b$ and its parent and still maintain the third
		property.
		By the implementation of swap stated in 1, the pointer to $b$ is still
		pointing to the same relative location in the heap.
		Notice that parent($b$) has a degree greater than $k+1$, and any
		possible new black node in the tree is of degree greater than $k+1$.
		Now, $b$ is a white node and parent($b$) is black. 
		Suppose parent($b$) is a root, then it needs to be coloured white in
		order to maintain property 2.
		After all these operations, there is no black node of degree $k$ with a
		parent of degree greater than $k+1$. 
		Hence, the new heap is a $k+1$-black$\&$white heap.
		(Note that $s$ might have a parent of degree $k+2$, this does not
		violate the fact that the new heap is a $k+1$-black$\&$white heap)

		By these three cases above, the $k$-black$\&$white heap H could be
		transformed into a $k+1$-black$\&$white heap.

		Since each operation in this algorithm runs in constant time and each
		line is executed by at most once, the algorithm runs in constant time.

	\item
	\begin{comment}
		Suppose you have a linked-list H of binomial trees that satisfies all
		the properties of a black$&$white heap except that it had one black node
		$b$ of degree $k$ whose parent has degree greater than $k+1$ and
		another black node of degree $k$ whose parent has degree $k+1$.
		Given pointers to $b$ and its sibling $s$ of degree $k+1$, explain how
		to transform H in constant time into a k'-black$&$white heap with the
		same set of nodes, for some $k' > k$.
	\end{comment}
		
		\begin{algorithmic}[1]
			\If{priority(parent($b$)) $\leq$ priority($b$)}
				\State colour $b$ white
			\ElsIf{$s$.firstchild is white}
				\If{$s$ is white}
					\State perform algorithm in part 2.
				\Else
					\State perform algorithm in part 3.
				\EndIf
			\Else
				\State perform algorithm in part 1. where $b = b$ and $b'
					= s$.firstchild
				\If{$b$ is black}
					\If{$s$ is white}
						\State perform algorithm in part 2.
					\Else
						\State perform algorithm in part 3.
					\EndIf
				\EndIf
			\EndIf
		\end{algorithmic}

		Suppose the priority of the parent of $b$ is less than or equal to
		priority of $b$, then $b$ could be coloured white and still holds the
		third property.
		After colouring, there would be only 1 black node of degree $k$, and
		no black node of degree $k$ with a parent of degree greater than $k+1$.
		The new heap is a $k+1$-black&white heap.

		Otherwise, first consider the case where the first child of $s$ is
		white. 
		Then, the algorithms in part 2 and part 3 could be performed here. 
		If $s$ is white, condition on $s$ in part 2 is met, and therefore,
		perform algorithm in part 2.
		Otherwise, perform algorithm in part 3.
		Either case, the resulting heap still has 1 node of degree $k$, while
		there is no node of degree $k$ with a parent of degree greater than
		$k+1$.
		The resulting heap will be a $k+1$-black&white heap.

		Now, consider when the first child of $s$ is black.
		Algorithm in part 2 and 3 could not be directly performed here.
		However, algorithm in part 1 could be performed as long as there are
		two nodes of the same degree. 
		Let the input for $b$ be $b$ and $b'$ be $s$.firstchild.

		There could be two possible outcomes: \\
		a. $s$.firschild is coloured white. \\
		b. $b$ is swapped with its parent. 

		Note that priority($b$) < priority(parent($b$)).

		Suppose a is the case, then there is only one node of degree $k$.
		But, $b$ is still a black node of degree $k$ with a parent of degree greater
		than $k+1$.
		However, this fall into the condition either in part 2 or 3.
		Depending on the colour of $s$, we can choose either of them to
		perform and the resulting heap would be a $k+1$-black&white heap.

		Consider case b.
		After the operation, $b$ is now a white node with degree $k$.
		There is only one black node of degree $k$, which is $s$.firstchild,
		and its parent $s$ is of degree $k+1$.
		Thus, the new heap is a $k+1$-black&white heap.

		Combine all those cases above, the resulting heap is always
		a $k+1$-black&white heap.

		Since algorithm 1, 2 and 3 all runs in constant time, each line in this
		algorithm runs in constant time. 
		Also, each line is executed by at most once in this algorithm. 
		Thus, the total run time for it is in constant time.

	\item
	\begin{comment}
		The operation DECREASE-KEY(H, x, v) takes a node $x$ in a black$&$white
		heap H and a value $v$ and decrease the priority of $x$ by $v$. 
		Explain how to perform DECREASE-KEY(H, x, v) in a black$&$white heap so
		that its amortized cost in $O(1)$.
		Use the number of black nodes in the black$&$white heap as your potential
		function or, equivalently, maintain the invariant that there is one
		token on each black node.
	\end{comment}
	
		\begin{algorithmic}[1]
			\Function{DECREASE-KEY}{$H,x,v$}
				\State priority($x$) $\gets$ priority($x$) $- v$
				\If{priority(parent($x$)) $>$ priority($x$)}
					\State
				\EndIf
			\EndFunction
		\end{algorithmic}

\end{enumerate}

\end{document}

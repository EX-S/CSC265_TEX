\documentclass[10pt]{article}
\usepackage{amsmath}
\usepackage{amssymb}
\usepackage{amsthm}
\usepackage{fullpage}
\usepackage{comment}
\usepackage{graphicx}
\usepackage{listings}
\usepackage{enumerate}
\usepackage{scrextend}
\usepackage{algorithm}
\usepackage{algpseudocode}

\newcommand{\iimplies}{\mbox{ IMPLIES }}
\newcommand{\oor}{\mbox{ OR }}
\newcommand{\aand}{\mbox{ AND }}
\newcommand{\nnot}{\mbox{ NOT }}
\newcommand{\iiff}{\mbox{ IFF }}
\newcommand{\xxor}{\mbox{ XOR }}

\begin{document}

\begin{center}
{\bf \Large \bf CSC265H Assignment 1}\\
\end{center}

\noindent
Yibin Zhao\\
1002996261\\

%Statement on collab.

\begin{enumerate}

\item
	The idea of my algorithm is is basically a BFS on this binary tree B that starts from the root and go down 1 level per iteration. If the current node is nil then stop searching down. If the current node is smaller than the number $x$, then add 1 to the counter, which keeps track of the number of nodes which are larger then $x$, and add its left child and right child to a list of nodes for the next iteration. If the counter exceeds $k$, then return true. If there is no nodes to be performed on the next iteration and the counter does not exceed $k$, return false. Otherwise, keep iterating.

\item 
\begin{algorithmic}[1]
	\Function{Tree-Compare}{$B,x,k$}
		\State $i \gets 0$
		\State $r \gets B$.root
		\State Queue $Q$ \Comment{Initialize a queue}
		\State \Call{ENQUEUE}{$Q, r$} \Comment{Add the root of the tree tothe queue}
		\While{$i < k$ and \Call{Empty}{$Q$} $==$ FALSE} \Comment{The loop halts when $i$ reaches $k$ or the queue is empty}
			\State $r \gets$ \Call{DEQUEUE}{$Q$}
			\If{$r \neq$ nil and $r$.key $> x$}
				\State \Call{ENQUEUE}{$Q, r$.left}
				\State \Call{ENQUEUE}{$Q, r$.right}
				\State $i \gets i+1$
			\EndIf
		\EndWhile
		\If{$i < k$}
			\State \Return False
		\Else
			\State \Return True
		\EndIf
	\EndFunction
\end{algorithmic}


\item %partial correctness
	The algorithm should return True if the $k$'th largest number in the tree is greater than $x$ and False otherwise. 
   
\begin{addmargin}[1em]{0em}
    \textbf{Proposition:} The $k$'th largest number in B is larger than $x
    \iff$ there at least exists $k$ numbers larger than $x$ in B.
    
    \begin{proof}
        Suppose the $k$'th largest number in B is larger than $x$. 
        \begin{addmargin}[1em]{0em}
            Then, the 1st to the $k-1$'th largest numbers are all greater than the $k$'th largest number in B, which are all greater than $x$. \\
            Thus, there at least exists $k$ numbers greater than $x$ in B. 
        \end{addmargin}
        Therefore, the $k$'th largest number in B is larger than $x \iimplies$ there at least exists $k$ number larger than $x$ in B. \\

        Suppose the $k$'th largest number in B is no greater than $x$.
        \begin{addmargin}[1em]{0em}
            Then, the $k+1$'th... are all no larger than $x$. \\
            Thus, there are at most $k-1$ numbers greater than $x$.
        \end{addmargin}
        Therefore, the $k$'th largest number in B is no greater than $x \iimplies$ there does not exist $k$ numbers larger than $x$ in B. \\
        By contrapositive, there at least exists $k$ numbers larger than $x$ in B $\iimplies$ the $k$'th largest number in B is larger than $x$. \\

        Above all, The $k$'th largest number in B is larger than $x \iff$ there at least exists $k$ numbers larger than $x$ in B. 
    \end{proof}
\end{addmargin}

\begin{addmargin}[1em]{0em}
    \textbf{Lemma 1:} Let $P=\{n_1,n_2, \cdots\}$ be the multiset of all nodes enqueued by the algorithm performed on $B$, $k$ and $x$, excluding nil.
    Then $P$ is a set.
    \begin{proof}
        B is a BST. \\
        By the properties of BST, there exists a unique path for each node from the root, since each node could only have 1 parent. \\
        If the root is nil, then the multiset $P$ is empty and thus $P$ is a set. \\
        Otherwise, by line 3 and 5, the root of B is enqueued into $Q$ before the first iteration of the while loop. \\
        Let $P(j)$ be the multiset of all nodes enqueued in $Q$ after iteration $j$ of the while loop. \\
        Let $Q(j)$ be the predicate that "P(j) is a set". \\

        We want to prove that $\forall j \in N, Q(j)$ \\
        \textbf{Base Case:} $j = 0$
        \begin{addmargin}[1em]{0em}
            By the proof above, $P(0) = \{ B.root \}$ and thus, $P(0)$ is a set. \\
            The base case $Q(0)$ is True.
        \end{addmargin}
        \textbf{Inductive Step:}
            Assume $Q(j)$ is satisfied for some $j \in \mathbb{N}$. \\
            Then, after iteration $j$ of the while loop, $P(j)$ is a set. \\
            Consider after executing line 7. \\
            \textit{Case 1:} $r =$ nil $\oor r.$key $\leq x$
            \begin{addmargin}[1em]{0em}
                The if statement on line 8 to 12 is not executed. \\
                Therefore, $P(j+1) = P(j)$. \\
                $Q(j)$ is true, so does $Q(j+1)$.
            \end{addmargin}
            \textit{Case 2:} $r \neq nil \aand r.$key $> x$
            \begin{addmargin}[1em]{0em}
               The condition on line 8 is satisfied and thus line 9 and 10 are executed. \\
               Since there is only 1 unique path to each node from the root, the path to r.left and r.right are unique. \\
               Also, $P(j)$ is a set, which means that $r$.left and $r$.right are not elements in $P(j)$, or otherwise, r is enqueued more than once. \\
               It doesn't matter if $r$.left or $r$.right is nil, since by definition of $P$, nil does not count as an element. \\
               Thus, after enqueue r.left and r.right, $P(j+1)$ is still a set. \\
               Hense, $Q(j+1)$.
            \end{addmargin}
            By these two cases, we can conclude that $Q(j+1)$ is true. \\
            Thus, $Q(j) \iimplies Q(j+1)$. \\
            Therefore, by induction, $\forall j \in \mathbb{N}, Q(j)$. \\
            Hense, $P$ is a set.
    \end{proof}
\end{addmargin}

The following is the proof of partial correctness:
\begin{proof}
    Let B be an arbitrary BST that satisfy the properties, and let $k \in \mathbb{N}$, $x \in \mathbb{R}$ be arbitrary. 
\begin{addmargin}[1em]{0em}
  Suppose the Algorithm returns True. \\
  By the if statement from line 14 to 18, $i \geq k$ after the while loop on line 6-13 halts. \\
  Before the while loop is performed, $i$ is initialized as 1 by line 2. \\
  Line 11 is the only excutation that could increment i, in which case, it must be performed for at least k times so that i exceeds k. \\
  Since line 11 belongs to the if statement from line 8 to 12, there are at least $k$ nodes which have the numbers larger than $x$ by Lemma 1. \\
  Thus, by the proposition stated above, the $k$'th largest number in B is larger than $x$. 
\end{addmargin}

Therefore, the Algotithem returns True $\iimplies$ the $k$'th largest number in B is larger than $x$. 

\begin{addmargin}[1em]{0em}
  Assume the algorithm returns False. \\
  Then, by the if statement from line 14 to line 18, $i < k$ after the while loop. \\
  By the conditions on line 6, the while loop halts when $i \geq k \oor$ \textbf{EMPTY}($Q$). \\
  Since, $i < k$, $Q$ is empty after the last iteration of the while loop. \\
  By Lemma 1, there are at least $i$ numbers in B larger than $x$. \\
  Suppose there are more than $i$ numbers in B larger than $x$. \\
  Then there at least exists a node, say $n$, not get enqueued by the algorithm. \\
  $n$ is not the root, since the root always get enqueued at line 5. \\
  Therefore $n$ has a parent. \\
  Let $S = \{n_0, n_1, \cdots, n_p\}$ be the set of nodes on the path from the root of B to $n$, where $n_p = n$, $n_0$ is the root and $p$ is some integer. \\
  Let $Q(j)$ be the predicate "$n_j \notin P \aand n_j$.key $> x$" 

  We want to prove $\forall j \in \{0, 1, \cdots, p\}. Q(j)$.  \\
  \textbf{Base Case:} $j = p$
  \begin{addmargin}[1em]{0em}
    By definition, $n_p = n$ and $n \notin P$. \\
    Also, $n$.key $> x$. \\
    Thus, $Q(p)$ holds.
  \end{addmargin}
  \textbf{Inductive Step:}
  \begin{addmargin}[1em]{0em}
    Assume $Q(j+1)$, where $j \in \{0, 1, \cdots , p-1 \}$. \\
    Then, $n_{j+1} \notin P \aand n_{j+1}$.key $> x$. \\
    Since $n_{j}$ is the parent of $n_{j+1}$, by the properties of $B$, the key of $n_{j}$ is greater than that of $n_{j+1}$, which is also greater than $x$. \\
    Suppose $n_{j} \in P$. \\
    \begin{addmargin}[1em]{0em}
      Since $n_{j}$.key $> x$, in some iteration where $r$ is assigned $n_{j}$, the if statement is executed. \\
      By line 9 and 10, $n_{j+1}$ must have been enqueued. \\
      It follows that $n_{j+1} \in P$, which is contradict to $Q(j+1)$.
    \end{addmargin}[1em]{0em}
    Therefore, $n_{j} \notin P$ and hence, $Q(j)$ is true.
  \end{addmargin}
  By induction, $\forall j \in \{0, 1, \cdots, p\}. Q(j)$. \\
  When $j = 0$, $n_0$ is the root of $B$ and we already know that the root is in $P$. \\
  This is a contradiction. \\
  Thus, there are only $i$ numbers in B greater than $x$. \\
  Hence, there are less than $k$ numbers in B greater than $x$. \\
  Equivalently, by proposition, the $k$'th largest number in B is no larger than $x$.
\end{addmargin}

Thus, the algorithm returns False $\iimplies$ the $k$'th largest number in B is no larger than $x$. \\
By contrapositive, the $k$'th largest number in B is larger than $x$ $\iimplies$ the algorithm returns True. 

Combine these two results above, the $k$'th largest number in B is larger than $x$ $\iiff$ the algorithm returns True. \\
Hence, the algorithm is partially correct.
\end{proof}

\item %O(k) time

\item %worst case lower bound

\end{enumerate}

\end{document}
